\documentclass{article}
\usepackage[utf8]{inputenc}
\usepackage{amsmath, amssymb}

\begin{document}

\subsection*{Bài 1.1.7. Giả sử các số tự nhiên mà mỗi số gồm ba chữ số $\overline{a_1 a_2 a_3}$, $\overline{b_1 b_2 b_3}$, $\overline{c_1 c_2 c_3}$ đều chia hết cho 17. Chứng minh rằng}
$$
\begin{vmatrix}
a_1 & a_2 & a_3 \\
b_1 & b_2 & b_3 \\
c_1 & c_2 & c_3
\end{vmatrix}
$$
\text{cũng chia hết cho 17.}

\section*{Giải.}

\text{Từ giả thiết $\overline{a_1 a_2 a_3}$ chia hết cho 17, ta suy ra $\overline{a_1 a_2 a_3} = 17k_1$ với $k_1 \in \mathbb{Z}$. Tương tự ta đặt $\overline{b_1 b_2 b_3} = 17k_2$, $\overline{c_1 c_2 c_3} = 17k_3$ với $k_2, k_3 \in \mathbb{Z}$.}

\text{Vậy}
$$
D = \begin{vmatrix}
a_1 & a_2 & a_3 \\
b_1 & b_2 & b_3 \\
c_1 & c_2 & c_3
\end{vmatrix}
= \begin{vmatrix}
a_1 & a_2 & 100a_1 + 10a_2 + a_3 \\
b_1 & b_2 & 100b_1 + 10b_2 + b_3 \\
c_1 & c_2 & 100c_1 + 10c_2 + c_3
\end{vmatrix}
$$
\text{(Thực hiện phép biến đổi cột: $C_3 \leftarrow C_3 + 100C_1 + 10C_2$. Định thức không đổi.)}
$$
= \begin{vmatrix}
a_1 & a_2 & \overline{a_1 a_2 a_3} \\
b_1 & b_2 & \overline{b_1 b_2 b_3} \\
c_1 & c_2 & \overline{c_1 c_2 c_3}
\end{vmatrix}
$$
\text{(Thay các giá trị theo giả thiết)}
$$
= \begin{vmatrix}
a_1 & a_2 & 17k_1 \\
b_1 & b_2 & 17k_2 \\
c_1 & c_2 & 17k_3
\end{vmatrix}
$$
\text{(Sử dụng tính chất: nhân một hàng/cột với một hằng số)}
$$
= 17 \begin{vmatrix}
a_1 & a_2 & k_1 \\
b_1 & b_2 & k_2 \\
c_1 & c_2 & k_3
\end{vmatrix} \quad \text{với } k_i \in \mathbb{N}, a_i, b_i, c_i \in \{0, \dots, 9\}.
$$
\text{Vậy }
$$
\begin{vmatrix}
a_1 & a_2 & a_3 \\
b_1 & b_2 & b_3 \\
c_1 & c_2 & c_3
\end{vmatrix}
$$
\text{cũng chia hết cho 17.}

\end{document}
