Thực hiện phép nhân các ma trận sau:
a) $\left(\begin{array}{ccc}1 & 3 & 2 \\ -4 & -1 & 5\end{array}\right)\left(\begin{array}{ccc}0 & 3 & 2 \\ -2 & -4 & 1 \\ 1 & 5 & -6\end{array}\right)$,
b) $\left(\begin{array}{ccc}-1 & 2 & 3 \\ 2 & 1 & -4\end{array}\right)\left(\begin{array}{cc}2 & -1 \\ 3 & 1 \\ 1 & 5\end{array}\right)\left(\begin{array}{cc}1 & 4 \\ 2 & 3\end{array}\right)$,
c) $\left(\begin{array}{cc}\cos \theta & -\sin \theta \\ \sin \theta & \cos \theta\end{array}\right)^n, n \in \mathbb{N}, n \geq 2$.

Giải. a) Ta có
$$
\left(\begin{array}{ccc}
1 & 3 & 2 \\
-4 & -1 & 5
\end{array}\right)\left(\begin{array}{ccc}
0 & 3 & 2 \\
-2 & -4 & 1 \\
1 & 5 & -6
\end{array}\right)=\left(\begin{array}{ccc}
-4 & 1 & -7 \\
7 & 17 & -39
\end{array}\right) .
$$

b) Tương tự ta có
$$
\begin{aligned}
& \left(\begin{array}{ccc}
-1 & 2 & 3 \\
2 & 1 & -4
\end{array}\right)\left(\begin{array}{cc}
2 & -1 \\
3 & 1 \\
1 & 5
\end{array}\right)\left(\begin{array}{cc}
1 & 4 \\
2 & 3
\end{array}\right) \\
& =\left(\begin{array}{cc}
7 & 18 \\
3 & -21
\end{array}\right)\left(\begin{array}{cc}
1 & 4 \\
2 & 3
\end{array}\right)=\left(\begin{array}{cc}
43 & 82 \\
-39 & -51
\end{array}\right) .
\end{aligned}
$$
c) Đặt $A=\left(\begin{array}{cc}\cos \theta & -\sin \theta \\ \sin \theta & \cos \theta\end{array}\right)$. Ta sẽ chứng minh đẳng thức sau bằng phương pháp quy nap
$$
A^n=\left(\begin{array}{cc}
\cos n \theta & -\sin n \theta \\
\sin n \theta & \cos n \theta
\end{array}\right)
$$
- Trước hết, ta có $A^2=A \cdot A=\left(\begin{array}{cc}\cos 2 \theta & -\sin 2 \theta \\ \sin 2 \theta & \cos 2 \theta\end{array}\right)$.
- Giả sừ ta có dẩng thức $A^k=\left(\begin{array}{cc}\cos k \theta & -\sin k \theta \\ \sin k \theta & \cos k \theta\end{array}\right)$.
- Ta sẽ chứng minh $A^{k+1}=\left(\begin{array}{cc}\cos (k+1) \theta & -\sin (k+1) \theta \\ \sin (k+1) \theta & \cos (k+1) \theta\end{array}\right)$.

Thật vây, áp dụng giả thiết quy nạp và thực hiện phép nhân ma trận ta có
$$
\begin{aligned}
A^{k+1}=A^k \cdot A & =\left(\begin{array}{cc}
\cos k \theta & -\sin k \theta \\
\sin k \theta & \cos k \theta
\end{array}\right)\left(\begin{array}{cc}
\cos \theta & -\sin \theta \\
\sin \theta & \cos \theta
\end{array}\right) \\
& =\left(\begin{array}{cc}
\cos (k+1) \theta & -\sin (k+1) \theta \\
\sin (k+1) \theta & \cos (k+1) \theta
\end{array}\right)
\end{aligned}
$$

Vậy $A^n=\left(\begin{array}{cc}\cos n \theta & -\sin n \theta \\ \sin n \theta & \cos n \theta\end{array}\right)$, vởi $n \in \mathbb{N}, n \geq 2$.
