Bài 1.1.3. Cho $A, B$ là hai ma trận vuông cùng cấp và giao hoán dược $(A B=B A)$. Chứng minh rằng
a) $(A+B)^2=A^2+2 A B+B^2$,
b) $A^2-B^2=(A-B)(A+B)$,
c) $(A+B)^n=\sum_{k=0}^n C_n^k A^{n-k} B^k$

Giải. a) $(A+B)(A+B)=A^2+A B+B A+B^2=A^2+2 A B+B^2$.
b) $(A-B)(A+B)=A^2+A B-B A-B^2=A^2-B^2$.
c) Ta sẽ chứng minh dẳng thức (*) bằng phương pháp quy nạp.
- Với $n=1$, dẳng thức hiển nhiên đúng.
- Với $n=2$, ta cũng có $(A+B)^2=A^2+2 A B+B^2$.
- Giả sử đẳng thức (*) dúng với $n=m$, nghĩa là
$$
\begin{aligned}
(A+B)^m & =\sum_{k=0}^m C_m^k A^{m-k} B^k \\
& =A^m+C_m^1 A^{m-1} B+C_m^2 A^{m-2} B^2+\cdots+C_m^{m-1} A B^{m-1}+B^m
\end{aligned}
$$
- Ta cần chứng minh (*) dúng với $n=m+1$. Thật vậy, ta có
$$
\begin{aligned}
& (A+B)^{m+1}=(A+B)^m(A+B) \\
= & \left(\sum_{k=0}^m C_m^k A^{m-k} B^k\right)(A+B) \\
= & \sum_{k=0}^m C_m^k A^{m+1-k} B^k+\sum_{k=0}^m C_m^k A^{m-k} B^{k+1} \\
= & C_m^0 A^{m+1}+\sum_{k=1}^m C_m^k A^{m+1-k} B^k+\sum_{k=1}^m C_m^{k-1} A^{m+1-k} B^k+C_m^m B^{m+1} \\
= & C_{m+1}^0 A^{m+1}+\sum_{k=1}^m\left(C_m^k+C_m^{k-1}\right) A^{m+1-k} B^k+C_{m+1}^{m+1} B^{m+1}
\end{aligned}
$$

Mặt khác, $C_m^k+C_m^{k-1}=C_{m+1}^k$, ta suy ra
$$
(A+B)^{m+1}=\sum_{k=0}^{m+1} C_{m+1}^k A^{m+1-k} B^k
$$

Vậy $(A+B)^n=\sum_{k=0}^n C_n^k A^{n-k} B^k$.
