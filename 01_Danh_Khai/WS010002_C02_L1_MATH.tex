Tìm các ma trận vuông cấp hai thỏa mãn
a) $A^2=O$,
b) $A^2=I$,
c) $A^2=A$,
d) $A B=B A$ với $B=\left(\begin{array}{ll}1 & 3 \\ 2 & 4\end{array}\right)$.

Giải. a) Đặt $A=\left(\begin{array}{ll}a & b \\ c & d\end{array}\right)$, khi đó $A^2=\left(\begin{array}{cc}a^2+b c & b(a+d) \\ c(a+d) & d^2+b c\end{array}\right)$. Ta có hệ phương trình
$$
\left\{\begin{array}{l}
a^2+b c=0 \\
d^2+b c=0 \\
b(a+d)=0 \\
c(a+d)=0
\end{array}\right.
$$
- Nếu $a+d=0$ thì $b, c$ tùy ý. Khi đó $a=\sqrt{-b c}$ hoặc $a=-\sqrt{-b c}$, (với $b c \leq 0$ ) nên lần lượt $d=-\sqrt{-b c}$ hoặc $d=\sqrt{-b c}$.
- Nếu $a+d \neq 0$ thì $b=0, c=0$. Từ đó suy ra $a=0, d=0$ (loại).

Vậy $A=\left(\begin{array}{cc}\sqrt{-b c} & b \\ c & -\sqrt{-b c}\end{array}\right)$ hoặc $A=\left(\begin{array}{cc}-\sqrt{-b c} & b \\ c & \sqrt{-b c}\end{array}\right)$ với $b c \leq 0$.
b) Tương tự câu a) ta có hệ phương trình
$$
\left\{\begin{array} { l } 
{ a ^ { 2 } + b c = 1 } \\
{ d ^ { 2 } + b c = 1 } \\
{ b ( a + d ) = 0 } \\
{ c ( a + d ) = 0 }
\end{array} \Leftrightarrow \left\{\begin{array}{l}
a^2+b c=1 \\
(a-d)(a+d)=0 \\
b(a+d)=0 \\
c(a+d)=0
\end{array}\right.\right.
$$
- Nếu $a+d=0$ thì $b, c$ tùy ý. Khi đó $a=\sqrt{1-b c}$ hoặc $a= -\sqrt{1-b c}$ (với $b c \leq 1$ ), tương ứng $d=-\sqrt{1-b c}$ hoặc $d= \sqrt{1-b c}$.
- Nếu $a+d \neq 0$ thì $b=0, c=0$. Khi đó $a=1$ hoặc $a=-1$, tương ứng $d=1$ hoặc $d=-1$.

Vậy
$$
A=\left[\begin{array}{ll}
I_2 & \\
-I_2 & \\
\left(\begin{array}{cc}
\sqrt{1-b c} & b \\
c & -\sqrt{1-b c}
\end{array}\right) & \text { với } b c \leq 1 \\
\left(\begin{array}{cc}
-\sqrt{1-b c} & b \\
c & \sqrt{1-b c}
\end{array}\right) & \text { với } b c \leq 1
\end{array}\right.
$$
c) Theo bài ra ta có hệ phương trình
$$
\left\{\begin{array} { l } 
{ a ^ { 2 } + b c = a } \\
{ d ^ { 2 } + b c = d } \\
{ b ( a + d ) = b } \\
{ c ( a + d ) = c }
\end{array} \Leftrightarrow \left\{\begin{array}{l}
a^2+b c=a \\
a-d=(a-d)(a+d) \\
b(a+d)=b \\
c(a+d)=c
\end{array}\right.\right.
$$
- Nếu $a+d=1$ thì $b, c$ tùy ý. Từ diều kiện $a^2+b c=a$, giải phương trình bậc hai ẩn $a$, ta suy ra $a=\frac{1+\sqrt{1-4 b c}}{2}$ hoặc $a=\frac{1-\sqrt{1-4 b c}}{2}$ với $b c \leq \frac{1}{4}$. Khi đó $d=1-a$.
- Nếu $a+d \neq 1$ thì $b=0, c=0$. Vậy $a=0$ hoặc $a=1$, tương ứng $d=0$ hoặc $d=1$.

Vậy ta thu clược
$$
A=\left[\begin{array}{lc}
O & \text { với } b c \leq \frac{1}{4} \\
I & \text { b } \\
\left(\frac{1+\sqrt{1-4 b c}}{2}\right. & \frac{1-\sqrt{1-4 b c}}{2} \\
c & b \\
\left(\frac{1-\sqrt{1-4 b c}}{2}\right. & \frac{1+\sqrt{1-4 b c}}{2} \\
c & \text { với } b c \leq \frac{1}{4}
\end{array} .\right.
$$
d) Ta có $A B=\left(\begin{array}{ll}a+2 b & 3 a+4 b \\ c+2 d & 3 c+4 d\end{array}\right), B A=\left(\begin{array}{cc}a+3 c & b+3 d \\ 2 a+4 c & 2 b+4 d\end{array}\right)$.

Tữ đó ta dược hệ phương trình
$$
\left\{\begin{array} { l } 
{ a + 3 c = a + 2 b } \\
{ b + 3 d = 3 a + 4 b } \\
{ 2 a + 4 c = c + 2 d } \\
{ 2 b + 4 d = 3 c + 4 d }
\end{array} \Leftrightarrow \left\{\begin{array} { l } 
{ 2 b = 3 c } \\
{ d = a + b } \\
{ 2 a + 3 c = 2 d }
\end{array} \Leftrightarrow \left\{\begin{array}{l}
2 b=3 c \\
d=a+b .
\end{array}\right.\right.\right.
$$

Vậy $A=\left(\begin{array}{cc}a & b \\ \frac{2 b}{3} & a+b\end{array}\right)$.
