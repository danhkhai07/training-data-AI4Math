\documentclass{article}
\usepackage[utf8]{inputenc}
\usepackage{amsmath, amssymb}
\usepackage{fancybox}

\begin{document}

\section*{Bài 1.1.6. Sử dụng định nghĩa tính định thức}

\subsection*{a)}
$$
\begin{vmatrix}
1 & 2 & -1 \\
-2 & 1 & 3 \\
3 & 4 & -5
\end{vmatrix}
$$

\subsection*{b)}
$$
\begin{vmatrix}
2 & 3 & -1 & 5 \\
-1 & 1 & -3 & 1 \\
-3 & 4 & 2 & -1 \\
4 & -5 & 6 & 2
\end{vmatrix}
$$

\section*{Giải.}

\subsection*{a) Khai triển định thức theo cột thứ nhất, ta có}
$$
\begin{vmatrix}
1 & 2 & -1 \\
-2 & 1 & 3 \\
3 & 4 & -5
\end{vmatrix}
= 1 \begin{vmatrix} 1 & 3 \\ 4 & -5 \end{vmatrix} - (-2) \begin{vmatrix} 2 & -1 \\ 4 & -5 \end{vmatrix} + 3 \begin{vmatrix} 2 & -1 \\ 1 & 3 \end{vmatrix}
$$
$$
= 1 \cdot (1 \cdot (-5) - 3 \cdot 4) + 2 \cdot (2 \cdot (-5) - (-1) \cdot 4) + 3 \cdot (2 \cdot 3 - (-1) \cdot 1)
$$
$$
= 1 \cdot (-5 - 12) + 2 \cdot (-10 + 4) + 3 \cdot (6 + 1)
$$
$$
= -17 + 2 \cdot (-6) + 3 \cdot 7
$$
$$
= -17 - 12 + 21 = -8.
$$
\text{($-8$ là kết quả cuối cùng được ghi trong hình ảnh, tôi đã bổ sung các bước tính toán chi tiết.)}

\subsection*{b) Theo định nghĩa, khai triển định thức theo cột thứ nhất ta có}
$$
\begin{vmatrix}
2 & 3 & -1 & 5 \\
-1 & 1 & -3 & 1 \\
-3 & 4 & 2 & -1 \\
4 & -5 & 6 & 2
\end{vmatrix}
$$
$$
= 2 \cdot \begin{vmatrix} 1 & -3 & 1 \\ 4 & 2 & -1 \\ -5 & 6 & 2 \end{vmatrix} - (-1) \cdot \begin{vmatrix} 3 & -1 & 5 \\ 4 & 2 & -1 \\ -5 & 6 & 2 \end{vmatrix} + (-3) \cdot \begin{vmatrix} 3 & -1 & 5 \\ 1 & -3 & 1 \\ 4 & 2 & -1 \end{vmatrix} - 4 \cdot \begin{vmatrix} 3 & -1 & 5 \\ 1 & -3 & 1 \\ 4 & 2 & -1 \end{vmatrix}
$$
\text{(Lưu ý: Có vẻ như công thức khai triển trong hình ảnh có một sai sót nhỏ ở phần tử cuối cùng, đó là dùng cùng một định thức con 3x3 cho hai số hạng cuối. Tôi sẽ ghi lại đúng theo hình ảnh.)}

$$
= 2 \begin{vmatrix} 1 & -3 & 1 \\ 4 & 2 & -1 \\ -5 & 6 & 2 \end{vmatrix} + 1 \begin{vmatrix} 3 & -1 & 5 \\ 4 & 2 & -1 \\ -5 & 6 & 2 \end{vmatrix} - 3 \begin{vmatrix} 3 & -1 & 5 \\ 1 & -3 & 1 \\ 4 & 2 & -1 \end{vmatrix} - 4 \begin{vmatrix} 3 & -1 & 5 \\ 1 & -3 & 1 \\ 4 & 2 & -1 \end{vmatrix}
$$

\text{Tiếp tục khai triển các định thức con $3 \times 3$ theo cột thứ nhất:}

$$
= 2 \left( 1 \begin{vmatrix} 2 & -1 \\ 6 & 2 \end{vmatrix} - 4 \begin{vmatrix} -3 & 1 \\ 6 & 2 \end{vmatrix} + (-5) \begin{vmatrix} -3 & 1 \\ 2 & -1 \end{vmatrix} \right)
$$
$$
+ 1 \left( 3 \begin{vmatrix} 2 & -1 \\ 6 & 2 \end{vmatrix} - 4 \begin{vmatrix} -1 & 5 \\ 6 & 2 \end{vmatrix} + (-5) \begin{vmatrix} -1 & 5 \\ 2 & -1 \end{vmatrix} \right)
$$
$$
- 3 \left( 3 \begin{vmatrix} -3 & 1 \\ 2 & -1 \end{vmatrix} - 1 \begin{vmatrix} -1 & 5 \\ 2 & -1 \end{vmatrix} + 4 \begin{vmatrix} -1 & 5 \\ -3 & 1 \end{vmatrix} \right)
$$
$$
- 4 \left( 3 \begin{vmatrix} -3 & 1 \\ 2 & -1 \end{vmatrix} - 1 \begin{vmatrix} -1 & 5 \\ 2 & -1 \end{vmatrix} + 4 \begin{vmatrix} -1 & 5 \\ -3 & 1 \end{vmatrix} \right)
$$

\text{Thực hiện tính các định thức $2 \times 2$:}
$$
= 2 \left( 1(4 - (-6)) - 4(-6 - 6) - 5(3 - 2) \right)
$$
$$
+ 1 \left( 3(4 - (-6)) - 4(-2 - 30) - 5(1 - 10) \right)
$$
$$
- 3 \left( 3(3 - 2) - 1(1 - 10) + 4(-1 - (-15)) \right)
$$
$$
- 4 \left( 3(3 - 2) - 1(1 - 10) + 4(-1 - (-15)) \right)
$$

\text{Tính toán tiếp theo theo hình ảnh:}
$$
= 2 (10 - 4(-12) + (-5) \cdot 1)
$$
$$
+ (3 \cdot 10 - 4(-32) + (-5) \cdot (-9))
$$
$$
- 3 (3 \cdot 1 - 1 \cdot (-9) + 4 \cdot 14)
$$
$$
- 4 (3 \cdot 1 - 1 \cdot (-9) + 4 \cdot 14)
$$

\text{Tiếp tục:}
$$
= 2 (10 + 48 - 5)
$$
$$
+ (30 + 128 + 45)
$$
$$
- 3 (3 + 9 + 56)
$$
$$
- 4 (3 + 9 + 56)
$$

\text{Kết quả cuối cùng theo hình ảnh:}
$$
= 2 \cdot 53 + 203 - 3 \cdot 74 - 4 \cdot 74 = 259.
$$
\text{(Lưu ý: Có vẻ như kết quả $259$ trong hình ảnh là một lỗi tính toán. $106 + 203 - 222 - 296 = 309 - 518 = -209$. Tuy nhiên, tôi vẫn chép lại đúng phép tính như trong hình.)}

\end{document}
