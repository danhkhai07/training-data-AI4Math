\documentclass{article}
\usepackage[utf8]{inputenc}
\usepackage{amsmath, amssymb}

\begin{document}

\section*{Problem}

\text{A food company produces 256 food products, all made from 91 basic ingredients.}
\text{The company wants to establish a simple data structure to quickly retrieve answers to the following questions:}
\begin{enumerate}
    \item \text{How many ingredients does a given product contain?}
    \item \text{How many products use a given pair of ingredients together?}
    \item \text{How many ingredients do two given products share?}
    \item \text{How many products use a certain ingredient?}
\end{enumerate}

\text{Specifically, the company requires the table structure to satisfy two conditions:}
\begin{enumerate}
    \item[(i)] \text{The answer to any of the above questions can be extracted easily and quickly (allowing matrix operations).}
    \item[(ii)] \text{If one of the 91 ingredients is added to or deleted from a product, only a single entry in the table needs to be changed.}
\end{enumerate}

\text{Is this feasible? Explain by setting up the matrix and interpreting the matrix products.}

\subsection*{Solution Setup}

\text{Yes, this is feasible using an incidence matrix $\mathbf{A}_{256 \times 91}$ where rows represent products and columns represent ingredients:}
$$
a_{ij} =
\begin{cases}
1 & \text{if Product } i \text{ contains Ingredient } j \\
0 & \text{otherwise}
\end{cases}
$$

\text{Condition (ii) is satisfied since changing one ingredient in one product requires changing only one entry $a_{ij}$.}

\text{Condition (i) is satisfied by the properties of matrix multiplication:}
\begin{itemize}
    \item \text{Question (c) (Ingredients shared by products $i$ and $k$) is given by the entry $(\mathbf{A}\mathbf{A}^T)_{ik}$.}
    \item \text{Question (b) (Products using ingredients $j$ and $k$ together) is given by the entry $(\mathbf{A}^T\mathbf{A})_{jk}$.}
\end{itemize}

\end{document}
