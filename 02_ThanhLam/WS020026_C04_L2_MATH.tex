\documentclass[a4paper,12pt]{article}
\usepackage[utf8]{inputenc}
\usepackage[vietnamese]{babel}
\usepackage{amsmath}
\usepackage{amsfonts}
\usepackage{amssymb}
\usepackage{geometry}

\geometry{left=2cm, right=2cm, top=2cm, bottom=2cm}

\begin{document}

\noindent \textbf{Bài 6.} Cho $\{v_1, v_2, \dots, v_m\}$ là hệ sinh của $W_1$, $\{u_1, u_2, \dots, u_n\}$ là hệ sinh của $W_2$ với $W_1, W_2$ là các không gian con của $V$. Chứng minh $\{v_1, \dots, v_m, u_1, \dots, u_n\}$ là hệ sinh của $W_1 + W_2$.

\section*{Lời giải}

Lấy một véc tơ bất kỳ $x \in W_1 + W_2$.
Theo định nghĩa của không gian tổng, véc tơ $x$ luôn phân tích được thành:
$$
x = x_1 + x_2 \quad \text{với } x_1 \in W_1, x_2 \in W_2.
$$

\begin{itemize}
    \item Vì $\{v_1, v_2, \dots, v_m\}$ là hệ sinh của $W_1$, nên $x_1$ biểu diễn được dưới dạng tổ hợp tuyến tính của các véc tơ $v_i$:
    $$
    x_1 = \sum_{i=1}^m k_i v_i = k_1 v_1 + k_2 v_2 + \dots + k_m v_m \quad (k_i \in \mathbb{R})
    $$
    
    \item Vì $\{u_1, u_2, \dots, u_n\}$ là hệ sinh của $W_2$, nên $x_2$ biểu diễn được dưới dạng tổ hợp tuyến tính của các véc tơ $u_j$:
    $$
    x_2 = \sum_{j=1}^n l_j u_j = l_1 u_1 + l_2 u_2 + \dots + l_n u_n \quad (l_j \in \mathbb{R})
    $$
\end{itemize}

Thay vào biểu thức của $x$, ta có:
$$
x = \left( \sum_{i=1}^m k_i v_i \right) + \left( \sum_{j=1}^n l_j u_j \right)
$$
$$
x = k_1 v_1 + \dots + k_m v_m + l_1 u_1 + \dots + l_n u_n
$$

Như vậy, mọi véc tơ $x \in W_1 + W_2$ đều biểu diễn được dưới dạng tổ hợp tuyến tính của hệ véc tơ hợp nhất $\{v_1, \dots, v_m, u_1, \dots, u_n\}$.

\textbf{Kết luận:} $\{v_1, \dots, v_m, u_1, \dots, u_n\}$ là hệ sinh của $W_1 + W_2$ (đpcm).

\end{document}