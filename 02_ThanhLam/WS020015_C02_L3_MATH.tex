\documentclass[a4paper,12pt]{article}
\usepackage[utf8]{inputenc}
\usepackage[vietnamese]{babel}
\usepackage{amsmath}
\usepackage{amsfonts}
\usepackage{amssymb}
\usepackage{geometry}

\geometry{left=2cm, right=2cm, top=2cm, bottom=2cm}

\begin{document}

\noindent \textbf{Bài 16.} Giải hệ phương trình sau:

\noindent
\begin{minipage}{0.3\textwidth}
a) $\begin{cases}
3x_1 - 5x_2 + 2x_3 + 4x_4 = 2 \\
7x_1 - 4x_2 + x_3 + 3x_4 = 5 \\
5x_1 + 7x_2 - 4x_3 - 6x_4 = 3
\end{cases}$
\end{minipage}
\hfill
\begin{minipage}{0.3\textwidth}
b) $\begin{cases}
3x_1 - x_2 + 3x_3 = 1 \\
-4x_1 + 2x_2 + x_3 = 3 \\
-2x_1 + x_2 + 4x_3 = 4 \\
10x_1 - 5x_2 - 6x_3 = -10
\end{cases}$
\end{minipage}
\hfill
\begin{minipage}{0.3\textwidth}
c) $\begin{cases}
2x_1 + 3x_2 + 4x_3 = 1 \\
3x_1 - x_2 + x_3 = 2 \\
5x_1 + 2x_2 + 5x_3 = 3 \\
x_1 - 4x_2 - 3x_3 = 1
\end{cases}$
\end{minipage}

\section*{Lời giải}

% --- Câu a ---
\subsection*{a)}
Lập ma trận hệ số mở rộng và biến đổi Gauss:
$$
\bar{A} = \left[ \begin{array}{cccc|c}
3 & -5 & 2 & 4 & 2 \\
7 & -4 & 1 & 3 & 5 \\
5 & 7 & -4 & -6 & 3
\end{array} \right]
\xrightarrow{\substack{3L_2 - 7L_1 \to L_2 \\ 3L_3 - 5L_1 \to L_3}}
\left[ \begin{array}{cccc|c}
3 & -5 & 2 & 4 & 2 \\
0 & 23 & -11 & -19 & 1 \\
0 & 46 & -22 & -38 & -1
\end{array} \right]
$$
$$
\xrightarrow{L_3 - 2L_2 \to L_3}
\left[ \begin{array}{cccc|c}
3 & -5 & 2 & 4 & 2 \\
0 & 23 & -11 & -19 & 1 \\
0 & 0 & 0 & 0 & -3
\end{array} \right]
$$
Ta thấy $r(A) = 2 \neq r(\bar{A}) = 3$ (vì hàng cuối có dạng $0= -3$ vô lý).
\\
$\Rightarrow$ \textbf{Hệ phương trình vô nghiệm.}

% --- Câu b ---
\subsection*{b)}
Lập ma trận mở rộng:
$$
\bar{A} = \left[ \begin{array}{ccc|c}
3 & -1 & 3 & 1 \\
-4 & 2 & 1 & 3 \\
-2 & 1 & 4 & 4 \\
10 & -5 & -6 & -10
\end{array} \right]
\xrightarrow{\substack{3L_2 + 4L_1 \to L_2 \\ 3L_3 + 2L_1 \to L_3 \\ 3L_4 - 10L_1 \to L_4}}
\left[ \begin{array}{ccc|c}
3 & -1 & 3 & 1 \\
0 & 2 & 15 & 13 \\
0 & 1 & 18 & 14 \\ % Note: Image calculation might simplify this step differently but logic holds
0 & -5 & -16 & -40 
\end{array} \right]
$$
Biến đổi tiếp (theo hình ảnh):
$$
\rightarrow \dots \rightarrow
\left[ \begin{array}{ccc|c}
3 & -1 & 3 & 1 \\
0 & 2 & 15 & 13 \\
0 & 0 & 21 & 15 \\
0 & 0 & 0 & 0
\end{array} \right]
$$
Ta có $r(A) = r(\bar{A}) = 3$ (= số ẩn). Hệ có nghiệm duy nhất.
Từ hàng 3: $21x_3 = 15 \Rightarrow x_3 = \frac{15}{21} = \frac{5}{7}$.
Thay ngược lên tìm $x_2, x_1$:
$$
\begin{cases}
3x_1 - x_2 + 3x_3 = 1 \\
2x_2 + 15x_3 = 13 \\
21x_3 = 15
\end{cases}
\Rightarrow (x_1, x_2, x_3) = \left( 0; \frac{8}{7}; \frac{5}{7} \right)
$$

% --- Câu c ---
\subsection*{c)}
Lập ma trận mở rộng:
$$
\bar{A} = \left[ \begin{array}{ccc|c}
2 & 3 & 4 & 1 \\
3 & -1 & 1 & 2 \\
5 & 2 & 5 & 3 \\
1 & -4 & -3 & 1
\end{array} \right]
\rightarrow \dots \rightarrow
\left[ \begin{array}{ccc|c}
2 & 3 & 4 & 1 \\
0 & -11 & -10 & 1 \\
0 & 0 & 0 & 0 \\
0 & 0 & 0 & 0
\end{array} \right]
$$
Ta có $r(A) = r(\bar{A}) = 2 < 3$ (số ẩn). Hệ có vô số nghiệm phụ thuộc vào $3-2=1$ tham số.
Hệ tương đương:
$$
\begin{cases}
2x_1 + 3x_2 + 4x_3 = 1 \\
-11x_2 - 10x_3 = 1
\end{cases}
$$
Đặt $x_3 = t$. Từ phương trình 2: $x_2 = \frac{-1-10t}{11}$.
Thay vào phương trình 1 tìm $x_1$:
$$
x_1 = \frac{1 - 3x_2 - 4x_3}{2} = \frac{3-14t}{22}
$$
Vậy nghiệm tổng quát:
$$
(x_1; x_2; x_3) = \left( \frac{3-14t}{22}; \frac{-1-10t}{11}; t \right), \quad t \in \mathbb{R}.
$$

\end{document}