\documentclass[a4paper,12pt]{article}
\usepackage[utf8]{inputenc}
\usepackage[vietnamese]{babel}
\usepackage{amsmath}
\usepackage{amsfonts}
\usepackage{amssymb}
\usepackage{geometry}

% Cấu hình lề trang
\geometry{
 a4paper,
 left=20mm,
 top=20mm,
}

\begin{document}

\section*{Bài 12. Tìm ma trận nghịch đảo của các ma trận sau:}

\noindent
\textbf{a)} $A = \begin{bmatrix} 3 & 4 \\ 5 & 7 \end{bmatrix}$ 
\hspace{2cm}
\textbf{b)} $B = \begin{bmatrix} 3 & -4 & 5 \\ 2 & -3 & 1 \\ 3 & -5 & 1 \end{bmatrix}$
\hspace{2cm}
\textbf{c)} $C = \begin{bmatrix} 1 & -a & 0 & 0 \\ 0 & 1 & -a & 0 \\ 0 & 0 & 1 & -a \\ 0 & 0 & 0 & 1 \end{bmatrix}$.

\section*{Lời giải}

% --- Phần a ---
\subsection*{a)}
Ta có: $\det A = 3(7) - 4(5) = 21 - 20 = 1 \neq 0$.

Ma trận nghịch đảo được tính như sau:
$$
A^{-1} = \frac{1}{\det A} \cdot \tilde{A}^T = \frac{1}{1} \begin{bmatrix} 7 & -4 \\ -5 & 3 \end{bmatrix} = \begin{bmatrix} 7 & -4 \\ -5 & 3 \end{bmatrix}
$$

% --- Phần b ---
\subsection*{b)}
Xét ma trận $B = \begin{bmatrix} 3 & -4 & 5 \\ 2 & -3 & 1 \\ 3 & -5 & 1 \end{bmatrix}$.

Ta tính được: $\det B = -3$.

Ma trận phụ hợp chuyển vị $\tilde{B}^T$ là:
$$
\tilde{B}^T = \begin{bmatrix} 2 & -21 & 11 \\ 1 & -12 & 7 \\ -1 & 3 & -1 \end{bmatrix}
$$
Suy ra:
$$
B^{-1} = \frac{1}{\det B} \cdot \tilde{B}^T = \frac{1}{-3} \begin{bmatrix} 2 & -21 & 11 \\ 1 & -12 & 7 \\ -1 & 3 & -1 \end{bmatrix} 
= \begin{bmatrix} -2/3 & 7 & -11/3 \\ -1/3 & 4 & -7/3 \\ 1/3 & -1 & 1/3 \end{bmatrix}.
$$

% --- Phần c ---
\subsection*{c)}
\textbf{Cách 1:} Ta có $\det C = 1$. Tìm $\tilde{C}^T \rightarrow C^{-1}$.

\textbf{Cách 2: Phương pháp Gauss-Jordan}

Thiết lập ma trận mở rộng $(C | I)$:
$$
\left[ \begin{array}{cccc|cccc}
1 & -a & 0 & 0 & 1 & 0 & 0 & 0 \\
0 & 1 & -a & 0 & 0 & 1 & 0 & 0 \\
0 & 0 & 1 & -a & 0 & 0 & 1 & 0 \\
0 & 0 & 0 & 1 & 0 & 0 & 0 & 1
\end{array} \right]
$$

Thực hiện biến đổi sơ cấp hàng $L_3 + aL_4 \rightarrow L_3$:
$$
\xrightarrow{L_3 + aL_4 \rightarrow L_3} 
\left[ \begin{array}{cccc|cccc}
1 & -a & 0 & 0 & 1 & 0 & 0 & 0 \\
0 & 1 & -a & 0 & 0 & 1 & 0 & 0 \\
0 & 0 & 1 & 0 & 0 & 0 & 1 & a \\
0 & 0 & 0 & 1 & 0 & 0 & 0 & 1
\end{array} \right]
$$

Tiếp tục biến đổi $L_2 + aL_3 \rightarrow L_2$:
$$
\xrightarrow{L_2 + aL_3 \rightarrow L_2} 
\left[ \begin{array}{cccc|cccc}
1 & -a & 0 & 0 & 1 & 0 & 0 & 0 \\
0 & 1 & 0 & 0 & 0 & 1 & a & a^2 \\
0 & 0 & 1 & 0 & 0 & 0 & 1 & a \\
0 & 0 & 0 & 1 & 0 & 0 & 0 & 1
\end{array} \right]
$$

Cuối cùng, biến đổi $L_1 + aL_2 \rightarrow L_1$ (khử phần tử $-a$ ở hàng 1):
$$
\rightarrow 
\left[ \begin{array}{cccc|cccc}
1 & 0 & 0 & 0 & 1 & a & a^2 & a^3 \\
0 & 1 & 0 & 0 & 0 & 1 & a & a^2 \\
0 & 0 & 1 & 0 & 0 & 0 & 1 & a \\
0 & 0 & 0 & 1 & 0 & 0 & 0 & 1
\end{array} \right]
$$

Vậy ma trận nghịch đảo là:
$$
\Rightarrow C^{-1} = \begin{bmatrix} 1 & a & a^2 & a^3 \\ 0 & 1 & a & a^2 \\ 0 & 0 & 1 & a \\ 0 & 0 & 0 & 1 \end{bmatrix}.
$$

\end{document}