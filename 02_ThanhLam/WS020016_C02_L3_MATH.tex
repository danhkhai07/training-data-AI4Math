\documentclass[a4paper,12pt]{article}
\usepackage[utf8]{inputenc}
\usepackage[vietnamese]{babel}
\usepackage{amsmath}
\usepackage{amsfonts}
\usepackage{amssymb}
\usepackage{geometry}

\geometry{left=2cm, right=2cm, top=2cm, bottom=2cm}

\begin{document}

\noindent \textbf{Bài 17.} Giải hệ phương trình sau bằng phương pháp Gauss:

\vspace{0.5cm}

% --- Câu a ---
\noindent \textbf{a)} (GK 20171)
$$
\begin{cases}
x + 2y - z + 3t = 12 \\
2x + 5y - z + 11t = 49 \\
3x + 6y - 4z + 13t = 49 \\
x + 2y - 2z + 9t = 33
\end{cases}
$$

\textbf{Lời giải:}

Lập ma trận hệ số mở rộng và biến đổi:
$$
\bar{A} = \left[ \begin{array}{cccc|c}
1 & 2 & -1 & 3 & 12 \\
2 & 5 & -1 & 11 & 49 \\
3 & 6 & 4 & 13 & 49 \\ % Lưu ý: Đề bài ảnh gốc là -4z, nhưng dòng 3 ma trận trong ảnh giải là 4 (có thể do in ấn mờ hoặc nhầm). Dựa vào tính toán L3 - 3L1 -> 0 0 -1, thì số hạng gốc phải là -4.
1 & 2 & -1 & 9 & 33
\end{array} \right]
\xrightarrow{\substack{L_2 - 2L_1 \to L_2 \\ L_3 - 3L_1 \to L_3 \\ L_4 - L_1 \to L_4}}
\left[ \begin{array}{cccc|c}
1 & 2 & -1 & 3 & 12 \\
0 & 1 & 1 & 5 & 25 \\
0 & 0 & -1 & 4 & 13 \\
0 & 0 & -1 & 6 & 21
\end{array} \right]
$$

Tiếp tục biến đổi khử hàng 4:
$$
\xrightarrow{L_4 - L_3 \to L_4}
\left[ \begin{array}{cccc|c}
1 & 2 & -1 & 3 & 12 \\
0 & 1 & 1 & 5 & 25 \\
0 & 0 & -1 & 4 & 13 \\
0 & 0 & 0 & 2 & 8
\end{array} \right]
$$
Ta có $r(A) = r(\bar{A}) = 4$ (= số ẩn). Hệ có nghiệm duy nhất.

Giải ngược từ dưới lên (đổi biến về $x_1, x_2, x_3, x_4$ theo lời giải mẫu):
$$
\begin{cases}
x_1 + 2x_2 - x_3 + 3x_4 = 12 \\
x_2 + x_3 + 5x_4 = 25 \\
-x_3 + 4x_4 = 13 \\
2x_4 = 8
\end{cases}
$$
Từ phương trình cuối $\Rightarrow x_4 = 4$. Thay dần lên trên ta được:
$$
\Rightarrow (x_1; x_2; x_3; x_4) = (-1; 2; 3; 4)
$$

\vspace{1cm}

% --- Câu b ---
\noindent \textbf{b)} (GK 20151)
$$
\begin{cases}
x + 2y + 3z + 4t = -4 \\
3x + 7y + 10z + 11t = -11 \\
x + 2y + 4z + 2t = -3 \\
x + 2y + 2z + 7t = -6
\end{cases}
$$

\textbf{Lời giải:}

Lập ma trận hệ số mở rộng:
$$
\bar{A} = \left[ \begin{array}{cccc|c}
1 & 2 & 3 & 4 & -4 \\
3 & 7 & 10 & 11 & -11 \\
1 & 2 & 4 & 2 & -3 \\
1 & 2 & 2 & 7 & -6
\end{array} \right]
\xrightarrow{\substack{L_2 - 3L_1 \to L_2 \\ L_3 - L_1 \to L_3 \\ L_4 - L_1 \to L_4}}
\left[ \begin{array}{cccc|c}
1 & 2 & 3 & 4 & -4 \\
0 & 1 & 1 & -1 & 1 \\
0 & 0 & 1 & -2 & 1 \\
0 & 0 & -1 & 3 & -2
\end{array} \right]
$$

Tiếp tục biến đổi:
$$
\xrightarrow{L_4 + L_3 \to L_4}
\left[ \begin{array}{cccc|c}
1 & 2 & 3 & 4 & -4 \\
0 & 1 & 1 & -1 & 1 \\
0 & 0 & 1 & -2 & 1 \\
0 & 0 & 0 & 1 & -1
\end{array} \right]
$$
Ta có $r(A) = r(\bar{A}) = 4$. Hệ có nghiệm duy nhất.

Giải hệ phương trình tam giác:
$$
\begin{cases}
x_1 + 2x_2 + 3x_3 + 4x_4 = -4 \\
x_2 + x_3 - x_4 = 1 \\
x_3 - 2x_4 = 1 \\
x_4 = -1
\end{cases}
\Rightarrow (x_1; x_2; x_3; x_4) = (1; 1; -1; -1)
$$

\end{document}