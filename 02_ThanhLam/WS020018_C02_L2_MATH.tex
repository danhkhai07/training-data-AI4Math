\documentclass[a4paper,12pt]{article}
\usepackage[utf8]{inputenc}
\usepackage[vietnamese]{babel}
\usepackage{amsmath}
\usepackage{amsfonts}
\usepackage{amssymb}

\begin{document}

\noindent \textbf{Bài 19 (CK 20172).} Tìm $m$ để hệ phương trình sau có nghiệm duy nhất:
$$
\begin{cases}
mx_1 + 2x_2 - x_3 = 3 \\
x_1 + mx_2 + 2x_3 = 4 \\
2x_1 + 3x_2 + x_3 = -m
\end{cases}
$$

\section*{Lời giải}

Hệ phương trình tuyến tính $AX=B$ (với $A$ là ma trận vuông cấp 3) có nghiệm duy nhất khi và chỉ khi $\det(A) \neq 0$.

Xét định thức của ma trận hệ số $A$:
$$
\det(A) = \begin{vmatrix}
m & 2 & -1 \\
1 & m & 2 \\
2 & 3 & 1
\end{vmatrix}
$$

Tính định thức (quy tắc Sarrus hoặc khai triển):
$$
\begin{aligned}
\det(A) &= m(m \cdot 1 - 2 \cdot 3) - 2(1 \cdot 1 - 2 \cdot 2) + (-1)(1 \cdot 3 - m \cdot 2) \\
&= m(m - 6) - 2(1 - 4) - (3 - 2m) \\
&= m^2 - 6m - 2(-3) - 3 + 2m \\
&= m^2 - 6m + 6 - 3 + 2m \\
&= m^2 - 4m + 3
\end{aligned}
$$
\textit{(Lưu ý: Trong hình ảnh gốc, tác giả dùng quy tắc đường chéo Sarrus để ra trực tiếp $m^2+5 - (4m+2)$)}

Để hệ có nghiệm duy nhất:
$$
\det(A) \neq 0 \iff m^2 - 4m + 3 \neq 0 \iff (m-1)(m-3) \neq 0 \iff \begin{cases} m \neq 1 \\ m \neq 3 \end{cases}
$$

Vậy với $m \neq 1$ và $m \neq 3$ thì hệ phương trình có nghiệm duy nhất.

\end{document}