\documentclass[a4paper,12pt]{article}
\usepackage[utf8]{inputenc}
\usepackage[vietnamese]{babel}
\usepackage{amsmath}
\usepackage{amsfonts}
\usepackage{amssymb}
\usepackage{geometry}

\geometry{left=2cm, right=2cm, top=2cm, bottom=2cm}

\begin{document}

\noindent \textbf{Bài 8.} Trong không gian $P_2[x]$, xét xem hệ véc tơ $B = \{u_1 = 1 + 2x, u_2 = 3x - x^2, u_3 = 2 - x + x^2\}$ độc lập tuyến tính hay phụ thuộc tuyến tính.

\section*{Lời giải}

Chọn cơ sở chính tắc của không gian $P_2[x]$ là $\mathcal{E} = \{1, x, x^2\}$.
Tọa độ của các véc tơ trong hệ $B$ đối với cơ sở $\mathcal{E}$ lần lượt là:
\begin{itemize}
    \item $u_1 = 1 + 2x + 0x^2 \Rightarrow [u_1]_{\mathcal{E}} = (1, 2, 0)^T$
    \item $u_2 = 0 + 3x - 1x^2 \Rightarrow [u_2]_{\mathcal{E}} = (0, 3, -1)^T$
    \item $u_3 = 2 - 1x + 1x^2 \Rightarrow [u_3]_{\mathcal{E}} = (2, -1, 1)^T$
\end{itemize}

Lập ma trận $A$ với các cột là tọa độ của các véc tơ $u_1, u_2, u_3$:
$$
A = \begin{bmatrix} 
1 & 0 & 2 \\ 
2 & 3 & -1 \\ 
0 & -1 & 1 
\end{bmatrix}
$$

Tính định thức của ma trận $A$:
$$
\det(A) = 1(3 \cdot 1 - (-1)(-1)) - 0 + 2(2(-1) - 3(0)) = 1(2) + 2(-2) = 2 - 4 = -2.
$$

Vì $\det(A) = -2 \neq 0$, nên hạng của hệ véc tơ bằng số chiều không gian (3).
\\
$\Rightarrow$ Hệ véc tơ $B$ \textbf{độc lập tuyến tính}.

\end{document}