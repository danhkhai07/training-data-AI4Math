\documentclass[a4paper,12pt]{article}
\usepackage[utf8]{inputenc}
\usepackage[vietnamese]{babel}
\usepackage{amsmath}
\usepackage{amsfonts}
\usepackage{amssymb}
\usepackage{geometry}

\geometry{left=2cm, right=2cm, top=2cm, bottom=2cm}

\begin{document}

\noindent \textbf{Bài 12 (CK 20151).} Trong $\mathbb{R}^4$, cho các véc tơ sau:
$$ u_1 = (1; 3; -2; 1), \quad u_2 = (-2; 3; 1; 1), \quad u_3 = (2; 1; 0; 1), \quad u = (1; -1; -3; m) $$
Tìm $m$ để $u \in \text{Span}\{u_1, u_2, u_3\}$.

\section*{Lời giải}

Để $u \in \text{Span}\{u_1, u_2, u_3\}$ thì véc tơ $u$ phải biểu diễn được dưới dạng tổ hợp tuyến tính của $u_1, u_2, u_3$. Tức là hệ phương trình sau (với ẩn $x_1, x_2, x_3$) phải có nghiệm:
$$ x_1 u_1 + x_2 u_2 + x_3 u_3 = u $$

Viết dưới dạng ma trận mở rộng $\bar{A}$:
$$
\bar{A} = \left[ \begin{array}{ccc|c}
1 & -2 & 2 & 1 \\
3 & 3 & 1 & -1 \\
-2 & 1 & 0 & -3 \\
1 & 1 & 1 & m
\end{array} \right]
$$

Thực hiện các phép biến đổi sơ cấp để đưa về dạng bậc thang:
\begin{itemize}
    \item $L_2 - 3L_1 \to L_2$
    \item $L_3 + 2L_1 \to L_3$
    \item $L_4 - L_1 \to L_4$
\end{itemize}

$$
\xrightarrow{}
\left[ \begin{array}{ccc|c}
1 & -2 & 2 & 1 \\
0 & 9 & -5 & -4 \\
0 & -3 & 4 & -1 \\
0 & 3 & -1 & m-1
\end{array} \right]
$$

Tiếp tục biến đổi khử cột 2:
\begin{itemize}
    \item Đổi chỗ $L_2$ và $L_3$ (để tính toán đơn giản hơn, hoặc giữ nguyên và nhân hệ số). Ta chọn cách nhân $3L_3 + L_2 \to L_3$ để khử phần tử $-3$.
    \item $3L_4 - L_2 \to L_4$ (để khử số $3$ ở dòng 4, lưu ý $3 \times 3 = 9$). Hoặc đơn giản hơn: $L_4 + L_3 \to L_4$ (lấy dòng 4 cộng dòng 3 hiện tại).
\end{itemize}

Ta lấy $L_4 + L_3$ (dòng 3 ở bước trên):
$$
\left[ \begin{array}{ccc|c}
1 & -2 & 2 & 1 \\
0 & 9 & -5 & -4 \\
0 & -3 & 4 & -1 \\
0 & 0 & 3 & m-2
\end{array} \right]
$$
\textit{(Lưu ý: Phép biến đổi trên chưa triệt để, ta nên làm chuẩn tắc như sau)}:

\textbf{Biến đổi chuẩn:}
Từ ma trận:
$$
\left[ \begin{array}{ccc|c}
1 & -2 & 2 & 1 \\
0 & -3 & 4 & -1 \\
0 & 9 & -5 & -4 \\
0 & 3 & -1 & m-1
\end{array} \right] \quad (\text{Đổi } L_2 \leftrightarrow L_3)
$$
$$
\xrightarrow{L_3 + 3L_2 \to L_3}
\left[ \begin{array}{ccc|c}
1 & -2 & 2 & 1 \\
0 & -3 & 4 & -1 \\
0 & 0 & 7 & -7 \\
0 & 3 & -1 & m-1
\end{array} \right]
$$
$$
\xrightarrow{L_4 + L_2 \to L_4}
\left[ \begin{array}{ccc|c}
1 & -2 & 2 & 1 \\
0 & -3 & 4 & -1 \\
0 & 0 & 7 & -7 \\
0 & 0 & 3 & m-2
\end{array} \right]
$$

Từ dòng 3: $7x_3 = -7 \Rightarrow x_3 = -1$.
Thay vào dòng 4: $3x_3 = m-2 \Rightarrow 3(-1) = m-2 \Rightarrow -3 = m-2 \Rightarrow m = -1$.

Để hệ có nghiệm (tức là $u$ thuộc Span) thì hạng của ma trận hệ số phải bằng hạng của ma trận mở rộng. Điều này tương đương với việc hệ phương trình tương thích.
Từ tính toán trên, ta thấy điều kiện là:
$$ m = -1 $$

\textbf{Kết luận:} Với $m = -1$ thì $u \in \text{Span}\{u_1, u_2, u_3\}$.

\end{document}