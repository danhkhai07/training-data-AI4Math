\documentclass[12pt,a4paper]{article}
\usepackage[utf8]{inputenc}
\usepackage[vietnamese]{babel}
\usepackage{amsmath}
\usepackage{amsfonts}
\usepackage{amssymb}
\usepackage{geometry}
\geometry{left=2cm, right=2cm, top=2cm, bottom=2cm}

\begin{document}

\noindent \textbf{Bài 7.} Không khai triển định thức mà dùng các tính chất của định thức để chứng minh:

\vspace{0.3cm}
\noindent a) $\begin{vmatrix} a_1+b_1x & a_1-b_1x & c_1 \\ a_2+b_2x & a_2-b_2x & c_2 \\ a_3+b_3x & a_3-b_3x & c_3 \end{vmatrix} = -2x \begin{vmatrix} a_1 & b_1 & c_1 \\ a_2 & b_2 & c_2 \\ a_3 & b_3 & c_3 \end{vmatrix}$ \hfill
b) $\begin{vmatrix} 1 & a & bc \\ 1 & b & ac \\ 1 & c & ab \end{vmatrix} = \begin{vmatrix} 1 & a & a^2 \\ 1 & b & b^2 \\ 1 & c & c^2 \end{vmatrix}.$

\begin{center}
    \textbf{Lời giải}
\end{center}

\noindent a)
$$
\begin{aligned}
&\begin{vmatrix} a_1+b_1x & a_1-b_1x & c_1 \\ a_2+b_2x & a_2-b_2x & c_2 \\ a_3+b_3x & a_3-b_3x & c_3 \end{vmatrix} = \begin{vmatrix} a_1+b_1x & 2a_1 & c_1 \\ a_2+b_2x & 2a_2 & c_2 \\ a_3+b_3x & 2a_3 & c_3 \end{vmatrix} (C_2+C_1 \rightarrow C_2) \\
&= 2 \begin{vmatrix} a_1+b_1x & a_1 & c_1 \\ a_2+b_2x & a_2 & c_2 \\ a_3+b_3x & a_3 & c_3 \end{vmatrix} = 2x \begin{vmatrix} b_1 & a_1 & c_1 \\ b_2 & a_2 & c_2 \\ b_3 & a_3 & c_3 \end{vmatrix} (C_1-C_2 \rightarrow C_1) \\
&= -2x \begin{vmatrix} a_1 & b_1 & c_1 \\ a_2 & b_2 & c_2 \\ a_3 & b_3 & c_3 \end{vmatrix}.
\end{aligned}
$$

\vspace{0.5cm}

\noindent b)
$$
\begin{aligned}
&\begin{vmatrix} 1 & a & bc \\ 1 & b & ac \\ 1 & c & ab \end{vmatrix} = \begin{vmatrix} 1 & a & a(a+b+c)+bc \\ 1 & b & b(a+b+c)+ac \\ 1 & c & c(a+b+c)+ab \end{vmatrix} (C_2 \times (a+b+c) + C_3 \rightarrow C_3) \\
&= \begin{vmatrix} 1 & a & a^2 \\ 1 & b & b^2 \\ 1 & c & c^2 \end{vmatrix} (-C_1 \times (ab+bc+ca) + C_3 \rightarrow C_3)
\end{aligned}
$$

\end{document}