\documentclass[a4paper,12pt]{article}
\usepackage[utf8]{inputenc}
\usepackage[vietnamese]{babel}
\usepackage{amsmath}
\usepackage{amsfonts}
\usepackage{amssymb}
\usepackage{geometry}

\geometry{left=2cm, right=2cm, top=2cm, bottom=2cm}

\begin{document}

\noindent \textbf{Bài 3.} Cho $V_1, V_2$ là hai không gian véc tơ con của không gian véc tơ $V$. Chứng minh:
\begin{enumerate}
    \item[a)] Giao của hai không gian con: $V_1 \cap V_2$ là không gian véc tơ con của $V$.
    \item[b)] Tổng của hai không gian con: $V_1 + V_2 = \{ u = u_1 + u_2 \mid u_1 \in V_1, u_2 \in V_2 \}$ là không gian véc tơ con của $V$.
\end{enumerate}

\section*{Lời giải}

Để chứng minh một tập hợp $W$ là không gian con của $V$, ta cần kiểm tra 2 điều kiện đóng kín:
\begin{itemize}
    \item $\forall u, v \in W \Rightarrow u + v \in W$.
    \item $\forall u \in W, \forall k \in \mathbb{R} \Rightarrow ku \in W$.
\end{itemize}

% --- Câu a ---
\subsection*{a) Chứng minh $V_1 \cap V_2$ là KGVT con}

\textbf{1. Tính đóng với phép cộng:}
Lấy hai véc tơ bất kỳ $u, v \in V_1 \cap V_2$.
$$
\Rightarrow \begin{cases} u, v \in V_1 \\ u, v \in V_2 \end{cases}
$$
Vì $V_1$ và $V_2$ đều là KGVT con, nên:
$$
\begin{cases} u + v \in V_1 \quad (\text{do } V_1 \text{ là KGVT con}) \\ u + v \in V_2 \quad (\text{do } V_2 \text{ là KGVT con}) \end{cases}
\Rightarrow u + v \in V_1 \cap V_2.
$$

\textbf{2. Tính đóng với phép nhân vô hướng:}
Lấy $u \in V_1 \cap V_2$ và số thực $k$ bất kỳ.
$$
\Rightarrow \begin{cases} u \in V_1 \\ u \in V_2 \end{cases}
\Rightarrow \begin{cases} ku \in V_1 \quad (\text{do } V_1 \text{ đóng kín với phép nhân}) \\ ku \in V_2 \quad (\text{do } V_2 \text{ đóng kín với phép nhân}) \end{cases}
$$
$$
\Rightarrow ku \in V_1 \cap V_2.
$$

Từ (1) và (2), suy ra $V_1 \cap V_2$ là một không gian véc tơ con của $V$.

% --- Câu b ---
\subsection*{b) Chứng minh $V_1 + V_2$ là KGVT con}

\textbf{1. Tính đóng với phép cộng:}
Lấy hai véc tơ bất kỳ $x, y \in V_1 + V_2$.
Theo định nghĩa, tồn tại $u_1, v_1 \in V_1$ và $u_2, v_2 \in V_2$ sao cho:
$$
x = u_1 + u_2 \quad \text{và} \quad y = v_1 + v_2
$$
Khi đó:
$$
x + y = (u_1 + u_2) + (v_1 + v_2) = (u_1 + v_1) + (u_2 + v_2)
$$
Do $V_1, V_2$ là KGVT con nên $(u_1 + v_1) \in V_1$ và $(u_2 + v_2) \in V_2$.
Vậy $x + y$ biểu diễn được dưới dạng tổng của một phần tử thuộc $V_1$ và một phần tử thuộc $V_2$.
$$
\Rightarrow x + y \in V_1 + V_2.
$$

\textbf{2. Tính đóng với phép nhân vô hướng:}
Lấy $x \in V_1 + V_2$ (với $x = u_1 + u_2$) và số thực $k$ bất kỳ.
$$
kx = k(u_1 + u_2) = ku_1 + ku_2
$$
Do $V_1, V_2$ là KGVT con nên $ku_1 \in V_1$ và $ku_2 \in V_2$.
Vậy $kx$ là tổng của một phần tử thuộc $V_1$ và một phần tử thuộc $V_2$.
$$
\Rightarrow kx \in V_1 + V_2.
$$

Kết luận: $V_1 + V_2$ là một không gian véc tơ con của $V$.

\end{document}