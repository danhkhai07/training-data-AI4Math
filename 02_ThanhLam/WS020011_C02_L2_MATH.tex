\documentclass{article}
\usepackage[utf8]{inputenc}
\usepackage[vietnamese]{babel}
\usepackage{amsmath}
\usepackage{amsfonts}
\usepackage{amssymb}

\begin{document}

\textbf{Bài 11 (GK20141).} Tìm $m$ để hạng của ma trận $A = \begin{bmatrix}
1 & -1 & 1 & 2 \\
-1 & 2 & 2 & 1 \\
1 & 0 & 4 & m
\end{bmatrix}$ bằng 2.

\begin{center}
    \textbf{Lời giải}
\end{center}

Ta thực hiện các phép biến đổi sơ cấp trên dòng để đưa ma trận về dạng bậc thang:

$$
A = \begin{bmatrix}
1 & -1 & 1 & 2 \\
-1 & 2 & 2 & 1 \\
1 & 0 & 4 & m
\end{bmatrix}
\xrightarrow{\substack{L_2 + L_1 \rightarrow L_2 \\ L_3 - L_1 \rightarrow L_3}}
\begin{bmatrix}
1 & -1 & 1 & 2 \\
0 & 1 & 3 & 3 \\
0 & 1 & 3 & m-2
\end{bmatrix}
$$

$$
\xrightarrow{L_3 - L_2 \rightarrow L_3}
\begin{bmatrix}
1 & -1 & 1 & 2 \\
0 & 1 & 3 & 3 \\
0 & 0 & 0 & m-5
\end{bmatrix}
$$

Để hạng của ma trận bằng 2 ($r(A)=2$), ma trận bậc thang phải có đúng 2 hàng khác 0. Điều này đồng nghĩa với việc hàng thứ 3 phải là hàng 0.

$$
\Rightarrow m - 5 = 0 \Longleftrightarrow m = 5
$$

Vậy với $m=5$ thì $r(A)=2$.

\end{document}