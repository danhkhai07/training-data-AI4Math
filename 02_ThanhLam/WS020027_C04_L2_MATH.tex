\documentclass[a4paper,12pt]{article}
\usepackage[utf8]{inputenc}
\usepackage[vietnamese]{babel}
\usepackage{amsmath}
\usepackage{amsfonts}
\usepackage{amssymb}
\usepackage{geometry}

\geometry{left=2cm, right=2cm, top=2cm, bottom=2cm}

\begin{document}

\noindent \textbf{Bài 7.} Trong $\mathbb{R}^3$, xét xem các hệ véc tơ sau độc lập tuyến tính hay phụ thuộc tuyến tính:
\begin{enumerate}
    \item[a)] $v_1 = (4; -2; 6), v_2 = (-6; 3; -9)$.
    \item[b)] $v_1 = (2; 3; -1), v_2 = (3; -1; 5), v_3 = (-1; 3; -4)$.
    \item[c)] $v_1 = (1; 2; 3), v_2 = (3; 6; 7), v_3 = (-3; 1; 3), v_4 = (0; 4; 2)$.
\end{enumerate}

\section*{Lời giải}

% --- Câu a ---
\subsection*{a)}
Xét hệ $\{v_1, v_2\}$ với $v_1 = (4; -2; 6)$ và $v_2 = (-6; 3; -9)$.
Ta thấy:
$$
\frac{-6}{4} = \frac{3}{-2} = \frac{-9}{6} = -\frac{3}{2}
$$
Suy ra $v_2 = -\frac{3}{2}v_1$.
Hai véc tơ tỷ lệ với nhau nên hệ $\{v_1, v_2\}$ \textbf{phụ thuộc tuyến tính}.

% --- Câu b ---
\subsection*{b)}
Xét hệ $\{v_1, v_2, v_3\}$. Lập định thức của ma trận tạo bởi các véc tơ này (xếp theo hàng hoặc cột):
$$
D = \det(A) = \begin{vmatrix}
2 & 3 & -1 \\
3 & -1 & 5 \\
-1 & 3 & -4
\end{vmatrix}
$$
Tính định thức:
$$
\begin{aligned}
D &= 2((-1)(-4) - 5(3)) - 3(3(-4) - 5(-1)) + (-1)(3(3) - (-1)(-1)) \\
&= 2(4 - 15) - 3(-12 + 5) - 1(9 - 1) \\
&= 2(-11) - 3(-7) - 8 \\
&= -22 + 21 - 8 \\
&= -9
\end{aligned}
$$
Vì $D = -9 \neq 0$, nên hạng của hệ véc tơ bằng 3.
$\Rightarrow$ Hệ $\{v_1, v_2, v_3\}$ \textbf{độc lập tuyến tính}.

% --- Câu c ---
\subsection*{c)}
Xét hệ gồm 4 véc tơ $\{v_1, v_2, v_3, v_4\}$ trong không gian $\mathbb{R}^3$.
Ta có định lý: "Trong không gian $n$ chiều, mọi hệ gồm $m$ véc tơ với $m > n$ đều phụ thuộc tuyến tính".
Ở đây:
\begin{itemize}
    \item Số chiều của không gian $\dim(\mathbb{R}^3) = 3$.
    \item Số lượng véc tơ trong hệ là $m = 4$.
\end{itemize}
Vì $4 > 3$ nên hệ véc tơ $\{v_1, v_2, v_3, v_4\}$ \textbf{phụ thuộc tuyến tính}.

\end{document}