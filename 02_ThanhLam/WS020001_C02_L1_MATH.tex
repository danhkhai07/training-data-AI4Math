\documentclass[12pt,a4paper]{article}
\usepackage[utf8]{inputenc}
\usepackage[vietnamese]{babel}
\usepackage{amsmath}
\usepackage{amsfonts}
\usepackage{amssymb}
\usepackage{geometry}
\geometry{left=2cm, right=2cm, top=2cm, bottom=2cm}

\begin{document}


% Đề bài
\noindent \textbf{Bài 1.} Cho các ma trận 
$A = \begin{bmatrix} 1 & -3 & 2 \\ 2 & 1 & -1 \\ 0 & 3 & -2 \end{bmatrix}, 
B = \begin{bmatrix} 2 & 1 & 1 \\ -2 & 3 & 0 \\ 1 & 2 & 4 \end{bmatrix}, 
C = \begin{bmatrix} -1 & 1 & 2 \\ 2 & 4 & -2 \end{bmatrix}.$

\vspace{0.5cm}

\noindent Trong các phép toán sau: $BC^T$, $A+BC$, $A^T B - C$, $A(BC)$, $(A+3B).C^T$, phép toán nào thực hiện được. Nếu thực hiện được cho biết kết quả.

% Lời giải
\begin{center}
    \textbf{Lời giải}
\end{center}

\noindent Các phép toán có thể thực hiện được là: $B.C^T$ ; $(A+3B).C^T$

% Phép tính 1
$$
B.C^T = \begin{bmatrix} 2 & 1 & 1 \\ -2 & 3 & 0 \\ 1 & 2 & 4 \end{bmatrix} \cdot \begin{bmatrix} -1 & 2 \\ 1 & 4 \\ 2 & -2 \end{bmatrix} = \begin{bmatrix} 1 & 6 \\ 5 & 8 \\ 9 & 2 \end{bmatrix}
$$

% Phép tính 2 - Bước trung gian
$$
A+3B = \begin{bmatrix} 1 & -3 & 2 \\ 2 & 1 & -1 \\ 0 & 3 & -2 \end{bmatrix} + 3\begin{bmatrix} 2 & 1 & 1 \\ -2 & 3 & 0 \\ 1 & 2 & 4 \end{bmatrix} = \begin{bmatrix} 7 & 0 & 5 \\ -4 & 10 & -1 \\ 3 & 9 & 10 \end{bmatrix}
$$

% Phép tính 2 - Kết quả cuối
$$
\Rightarrow (A+3B).C^T = \begin{bmatrix} 7 & 0 & 5 \\ -4 & 10 & -1 \\ 3 & 9 & 10 \end{bmatrix} \cdot \begin{bmatrix} -1 & 2 \\ 1 & 4 \\ 2 & -2 \end{bmatrix} = \begin{bmatrix} 3 & 4 \\ 12 & 34 \\ 26 & 22 \end{bmatrix}
$$

\end{document}