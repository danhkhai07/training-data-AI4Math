\documentclass[a4paper,12pt]{article}
\usepackage[utf8]{inputenc}
\usepackage[vietnamese]{babel}
\usepackage{amsmath}
\usepackage{amsfonts}
\usepackage{amssymb}
\usepackage{geometry}

\geometry{left=2cm, right=2cm, top=2cm, bottom=2cm}

\begin{document}

\noindent \textbf{Bài 2.} Chứng minh các tập hợp con của các không gian véc tơ quen thuộc sau là các không gian véc tơ con của chúng:
\begin{enumerate}
    \item[a)] Tập $E = \{(x_1, x_2, x_3) \in \mathbb{R}^3 \mid 2x_1 - 5x_2 + 3x_3 = 0\}$.
    \item[b)] Tập các đa thức có hệ số bậc nhất bằng 0 (hệ số của $x$) của KGVT $P_n[x]$.
    \item[c)] Tập các ma trận tam giác trên của các ma trận vuông cấp $n$.
    \item[d)] Tập các ma trận đối xứng của tập các ma trận vuông cấp $n$.
    \item[e)] Tập các ma trận phản xứng của tập các ma trận vuông cấp $n$ ($a_{ij} = -a_{ji}$).
\end{enumerate}

\section*{Lời giải}

Để chứng minh tập con $W \subset V$ ($W \neq \emptyset$) là một không gian véc tơ con của $V$, ta cần kiểm tra 2 điều kiện đóng kín:
\begin{itemize}
    \item $\forall u, v \in W \Rightarrow u + v \in W$.
    \item $\forall u \in W, \forall k \in \mathbb{R} \Rightarrow ku \in W$.
\end{itemize}

% --- Câu a ---
\subsection*{a)}
Xét tập $E = \{(x_1, x_2, x_3) \in \mathbb{R}^3 \mid 2x_1 - 5x_2 + 3x_3 = 0\}$.
\begin{itemize}
    \item Rõ ràng vectơ không $\mathbf{0} = (0, 0, 0) \in E$ vì $2(0) - 5(0) + 3(0) = 0$.
    \item Lấy $u = (x_1, x_2, x_3) \in E$ và $v = (y_1, y_2, y_3) \in E$. Ta có:
    $$
    2x_1 - 5x_2 + 3x_3 = 0 \quad \text{và} \quad 2y_1 - 5y_2 + 3y_3 = 0
    $$
    Xét tổng $u + v = (x_1+y_1, x_2+y_2, x_3+y_3)$. Kiểm tra điều kiện:
    $$
    2(x_1+y_1) - 5(x_2+y_2) + 3(x_3+y_3) = (2x_1 - 5x_2 + 3x_3) + (2y_1 - 5y_2 + 3y_3) = 0 + 0 = 0
    $$
    $\Rightarrow u + v \in E$.
    \item Xét tích vô hướng $ku = (kx_1, kx_2, kx_3)$ với $k \in \mathbb{R}$:
    $$
    2(kx_1) - 5(kx_2) + 3(kx_3) = k(2x_1 - 5x_2 + 3x_3) = k \cdot 0 = 0
    $$
    $\Rightarrow ku \in E$.
\end{itemize}
Vậy $E$ là KGVT con của $\mathbb{R}^3$.

% --- Câu b ---
\subsection*{b)}
Gọi $W$ là tập các đa thức thuộc $P_n[x]$ có hệ số của $x$ bằng 0.
Mọi đa thức $p(x) \in W$ có dạng: $p(x) = a_0 + 0x + a_2x^2 + \dots + a_nx^n$.
\begin{itemize}
    \item Với $p(x), q(x) \in W$, hệ số bậc 1 của chúng đều bằng 0. Khi cộng hai đa thức, hệ số bậc 1 của tổng bằng tổng các hệ số bậc 1: $0 + 0 = 0$.
    $\Rightarrow p(x) + q(x) \in W$.
    \item Với $p(x) \in W$ và $k \in \mathbb{R}$, hệ số bậc 1 của $kp(x)$ là $k \cdot 0 = 0$.
    $\Rightarrow kp(x) \in W$.
\end{itemize}
Vậy $W$ là KGVT con của $P_n[x]$.

% --- Câu c ---
\subsection*{c)}
Gọi $U$ là tập các ma trận tam giác trên cấp $n$. Ma trận $A = [a_{ij}]$ là tam giác trên nếu $a_{ij} = 0$ với mọi $i > j$.
\begin{itemize}
    \item Tổng hai ma trận tam giác trên là một ma trận tam giác trên (các phần tử dưới đường chéo chính đều bằng tổng của các số 0).
    \item Tích một số với ma trận tam giác trên vẫn là ma trận tam giác trên (các số 0 được nhân với $k$ vẫn là 0).
\end{itemize}
Vậy tập các ma trận tam giác trên là KGVT con của không gian các ma trận vuông cấp $n$.

% --- Câu d ---
\subsection*{d)}
Gọi $S$ là tập các ma trận đối xứng ($A^T = A$).
\begin{itemize}
    \item Cho $A, B \in S$. Ta có $(A+B)^T = A^T + B^T = A + B$. Vậy $A+B \in S$.
    \item Cho $A \in S, k \in \mathbb{R}$. Ta có $(kA)^T = k(A^T) = kA$. Vậy $kA \in S$.
\end{itemize}
Vậy tập các ma trận đối xứng là KGVT con.

% --- Câu e ---
\subsection*{e)}
Gọi $K$ là tập các ma trận phản xứng ($A^T = -A$).
\begin{itemize}
    \item Cho $A, B \in K$. Ta có $(A+B)^T = A^T + B^T = (-A) + (-B) = -(A+B)$.
    $\Rightarrow A+B \in K$.
    \item Cho $A \in K, k \in \mathbb{R}$. Ta có $(kA)^T = k(A^T) = k(-A) = -(kA)$.
    $\Rightarrow kA \in K$.
\end{itemize}
Vậy tập các ma trận phản xứng là KGVT con.

\end{document}