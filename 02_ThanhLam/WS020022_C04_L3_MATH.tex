\documentclass[a4paper,12pt]{article}
\usepackage[utf8]{inputenc}
\usepackage[vietnamese]{babel}
\usepackage{amsmath}
\usepackage{amsfonts}
\usepackage{amssymb}
\usepackage{geometry}

\geometry{left=2cm, right=2cm, top=2cm, bottom=2cm}

\begin{document}

\noindent \textbf{Bài 11.} Trong $P_3[x]$ cho các véc tơ:
$$v_1 = 1, \quad v_2 = 1+x, \quad v_3 = x+x^2, \quad v_4 = x^2+x^3$$
\begin{enumerate}
    \item[a)] Chứng minh $B = \{v_1, v_2, v_3, v_4\}$ là một cơ sở của $P_3[x]$.
    \item[b)] Tìm toạ độ của véc tơ $v = 2 + 3x - x^2 + 2x^3$ đối với cơ sở trên.
    \item[c)] Tìm tọa độ của véc tơ $v = a_0 + a_1x + a_2x^2 + a_3x^3$ đối với cơ sở trên.
\end{enumerate}

\section*{Lời giải}

Chọn cơ sở chính tắc của $P_3[x]$ là $\mathcal{E} = \{1, x, x^2, x^3\}$.

\subsection*{a) Chứng minh cơ sở}
Ma trận các véc tơ của hệ $B$ đối với cơ sở $\mathcal{E}$ (xếp theo cột):
\begin{itemize}
    \item $v_1 = 1 \Rightarrow [v_1]_\mathcal{E} = (1, 0, 0, 0)^T$
    \item $v_2 = 1+x \Rightarrow [v_2]_\mathcal{E} = (1, 1, 0, 0)^T$
    \item $v_3 = x+x^2 \Rightarrow [v_3]_\mathcal{E} = (0, 1, 1, 0)^T$
    \item $v_4 = x^2+x^3 \Rightarrow [v_4]_\mathcal{E} = (0, 0, 1, 1)^T$
\end{itemize}

Ma trận chuyển cơ sở $P_{\mathcal{E} \to B}$:
$$
P = \begin{bmatrix}
1 & 1 & 0 & 0 \\
0 & 1 & 1 & 0 \\
0 & 0 & 1 & 1 \\
0 & 0 & 0 & 1
\end{bmatrix}
$$
Ta thấy $\det(P) = 1 \cdot 1 \cdot 1 \cdot 1 = 1 \neq 0$.
Do đó hệ $B$ độc lập tuyến tính. Vì $\dim(P_3[x]) = 4$, suy ra $B$ là một cơ sở của $P_3[x]$.

\subsection*{b) Tìm tọa độ vectơ $v = 2 + 3x - x^2 + 2x^3$}
Tọa độ của $v$ trong cơ sở chính tắc: $[v]_\mathcal{E} = (2; 3; -1; 2)^T$.
Công thức đổi tọa độ: $[v]_B = P^{-1} \cdot [v]_\mathcal{E}$.

Tìm $P^{-1}$ (bằng phương pháp giải hệ hoặc ma trận phụ hợp):
$$
P^{-1} = \begin{bmatrix}
1 & -1 & 1 & -1 \\
0 & 1 & -1 & 1 \\
0 & 0 & 1 & -1 \\
0 & 0 & 0 & 1
\end{bmatrix}
$$
Khi đó:
$$
[v]_B = \begin{bmatrix}
1 & -1 & 1 & -1 \\
0 & 1 & -1 & 1 \\
0 & 0 & 1 & -1 \\
0 & 0 & 0 & 1
\end{bmatrix} \cdot \begin{bmatrix} 2 \\ 3 \\ -1 \\ 2 \end{bmatrix}
= \begin{bmatrix} 2 - 3 - 1 - 2 \\ 0 + 3 + 1 + 2 \\ 0 + 0 - 1 - 2 \\ 0 + 0 + 0 + 2 \end{bmatrix}
= \begin{bmatrix} -4 \\ 6 \\ -3 \\ 2 \end{bmatrix}
$$

\subsection*{c) Tìm tọa độ vectơ tổng quát $v = a_0 + a_1x + a_2x^2 + a_3x^3$}
Tọa độ trong cơ sở chính tắc: $[v]_\mathcal{E} = (a_0; a_1; a_2; a_3)^T$.
Áp dụng ma trận nghịch đảo $P^{-1}$ đã tìm được ở câu b:
$$
[v]_B = P^{-1} [v]_\mathcal{E} = \begin{bmatrix}
1 & -1 & 1 & -1 \\
0 & 1 & -1 & 1 \\
0 & 0 & 1 & -1 \\
0 & 0 & 0 & 1
\end{bmatrix} \cdot \begin{bmatrix} a_0 \\ a_1 \\ a_2 \\ a_3 \end{bmatrix}
= \begin{bmatrix} a_0 - a_1 + a_2 - a_3 \\ a_1 - a_2 + a_3 \\ a_2 - a_3 \\ a_3 \end{bmatrix}
$$

\end{document}