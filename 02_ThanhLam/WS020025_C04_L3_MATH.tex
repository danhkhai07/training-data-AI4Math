\documentclass[a4paper,12pt]{article}
\usepackage[utf8]{inputenc}
\usepackage[vietnamese]{babel}
\usepackage{amsmath}
\usepackage{amsfonts}
\usepackage{amssymb}
\usepackage{geometry}

\geometry{left=2cm, right=2cm, top=2cm, bottom=2cm}

\begin{document}

\noindent \textbf{Bài 5.} Trong không gian véc tơ $V$, cho hệ véctơ $\{u_1, u_2, \dots, u_n, u_{n+1}\}$ là phụ thuộc tuyến tính và $\{u_1, u_2, \dots, u_n\}$ là hệ độc lập tuyến tính. Chứng minh $u_{n+1}$ là tổ hợp tuyến tính của các véc tơ $u_1, u_2, \dots, u_n$.

\section*{Lời giải}

Theo giả thiết, hệ $\{u_1, u_2, \dots, u_n, u_{n+1}\}$ phụ thuộc tuyến tính.
Theo định nghĩa, tồn tại các hệ số thực $k_1, k_2, \dots, k_{n+1}$ không đồng thời bằng 0 sao cho:
\begin{equation} \label{eq:1}
k_1 u_1 + k_2 u_2 + \dots + k_n u_n + k_{n+1} u_{n+1} = \theta
\end{equation}

Ta xét hệ số $k_{n+1}$:
\begin{itemize}
    \item \textbf{Trường hợp 1:} Giả sử $k_{n+1} = 0$.
    Khi đó phương trình (\ref{eq:1}) trở thành:
    $$
    k_1 u_1 + k_2 u_2 + \dots + k_n u_n = \theta
    $$
    Do các hệ số $k_i$ không đồng thời bằng 0, nên trong các số $k_1, \dots, k_n$ phải có ít nhất một số khác 0.
    Điều này suy ra hệ $\{u_1, u_2, \dots, u_n\}$ phụ thuộc tuyến tính.
    $\Rightarrow$ Mâu thuẫn với giả thiết hệ $\{u_1, \dots, u_n\}$ là độc lập tuyến tính.
    Vậy trường hợp này không xảy ra.
    
    \item \textbf{Trường hợp 2:} Do đó, ta phải có $k_{n+1} \neq 0$.
    Từ phương trình (\ref{eq:1}), ta chuyển vế:
    $$
    k_{n+1} u_{n+1} = -k_1 u_1 - k_2 u_2 - \dots - k_n u_n
    $$
    Do $k_{n+1} \neq 0$, ta có thể chia hai vế cho $k_{n+1}$:
    $$
    u_{n+1} = \left( -\frac{k_1}{k_{n+1}} \right)u_1 + \left( -\frac{k_2}{k_{n+1}} \right)u_2 + \dots + \left( -\frac{k_n}{k_{n+1}} \right)u_n
    $$
    Đặt $\alpha_i = -\frac{k_i}{k_{n+1}}$, ta có:
    $$
    u_{n+1} = \alpha_1 u_1 + \alpha_2 u_2 + \dots + \alpha_n u_n
    $$
\end{itemize}

Vậy $u_{n+1}$ biểu diễn được dưới dạng tổ hợp tuyến tính của $u_1, u_2, \dots, u_n$ (đpcm).

\end{document}