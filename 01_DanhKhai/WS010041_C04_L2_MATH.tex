\documentclass{article}
\usepackage[utf8]{inputenc}
\usepackage{amsmath, amssymb}

\begin{document}

\subsection*{Cho $A = \{ f: [a, b] \to \mathbb{R} \mid f(x) \ge 0, \forall x \in [a, b] \}$. Chứng minh $A$ không phải là không gian con của không gian tuyến tính $C_{[a, b]}$ (các hàm số liên tục trên $[a, b]$).}

\section*{Giải.}

\text{Để chứng minh $A$ không phải là không gian con của $C_{[a, b]}$, ta chỉ cần chỉ ra một trong hai điều kiện đóng của không gian con bị vi phạm.}

\text{Tập $A$ là tập hợp các hàm liên tục nhận giá trị không âm trên đoạn $[a, b]$.}
\text{Phép cộng các hàm số và nhân hàm số với một số thực là các phép toán thông thường.}

\text{Điều kiện cần cho $A$ là không gian con là $A$ phải đóng với phép nhân vô hướng. }

\text{Ta xét một hàm số $f(x) \in A$ bất kỳ, nghĩa là $f(x) \ge 0, \forall x \in [a, b]$.}

\text{Chọn số vô hướng $\alpha = -1 \in \mathbb{R}$.}

\text{Ta xét hàm số $g(x) = \alpha f(x) = -f(x)$.}

\text{Vì $f(x) \ge 0$, nên $g(x) = -f(x) \le 0, \forall x \in [a, b]$.}

\text{Nếu $f(x)$ không phải là hàm zero (tức là tồn tại $x_0$ sao cho $f(x_0) > 0$), thì $g(x_0) = -f(x_0) < 0$.}

\text{Theo định nghĩa của tập $A$, hàm số $g(x)$ không thỏa mãn điều kiện $g(x) \ge 0$ (trừ khi $f(x) \equiv 0$).}

\text{Do đó, $g(x) = -f(x) \notin A$ (trừ khi $f(x)$ là hàm zero).}

\text{Vậy, $A$ không đóng với phép nhân vô hướng. $A$ không phải là không gian con của $C_{[a, b]}$.}

\end{document}
