\documentclass{article}
\usepackage[utf8]{inputenc}
\usepackage{amsmath, amssymb}
\usepackage{cases}

\begin{document}

\subsection*{Bài 2.1.32. Hãy tìm chiều và chỉ ra một cơ sở của không gian tuyến tính $A = \{(x_1, x_2, x_3, x_4) \in \mathbb{R}^4 \mid x_1+x_2+x_3+x_4 = 0\}$.}

\section*{Giải.}

\text{Với mỗi } $\mathbf{u} = (x_1, x_2, x_3, x_4) \in A \text{ bất kỳ, ta có } x_4 = -x_1 - x_2 - x_3$. \text{Ta viết véctơ } $\mathbf{u}$ \text{ dưới dạng tổ hợp tuyến tính:}
$$
\mathbf{u} = (x_1, x_2, x_3, -x_1 - x_2 - x_3)
$$
$$
= x_1(1, 0, 0, -1) + x_2(0, 1, 0, -1) + x_3(0, 0, 1, -1).
$$
\text{Đặt } $\mathbf{u}_1 = (1, 0, 0, -1), \mathbf{u}_2 = (0, 1, 0, -1), \mathbf{u}_3 = (0, 0, 1, -1)$, \text{ ta suy ra } $\mathbf{u} = x_1\mathbf{u}_1 + x_2\mathbf{u}_2 + x_3\mathbf{u}_3$. \text{Do đó } $\{\mathbf{u}_1, \mathbf{u}_2, \mathbf{u}_3\}$ \text{ là một hệ sinh của } $A$.

\text{Mặt khác, xét } $\alpha\mathbf{u}_1 + \beta\mathbf{u}_2 + \gamma\mathbf{u}_3 = \mathbf{0}$
$$
\Leftrightarrow \alpha(1, 0, 0, -1) + \beta(0, 1, 0, -1) + \gamma(0, 0, 1, -1) = (0, 0, 0, 0).
$$
\text{Ta thu được } $\alpha = \beta = \gamma = 0$ \text{ là nghiệm duy nhất thỏa mãn phương trình trên. Do đó } $\{\mathbf{u}_1, \mathbf{u}_2, \mathbf{u}_3\}$ \text{ là một hệ véctơ độc lập tuyến tính. Vậy } $\{\mathbf{u}_1, \mathbf{u}_2, \mathbf{u}_3\}$ \text{ là một cơ sở của không gian } $A$ \text{ và } $\dim A = 3$.

\end{document}
