\documentclass{article}
\usepackage[utf8]{inputenc}
\usepackage{amsmath, amssymb}
\usepackage{cases}

\begin{document}

\subsection*{Bài 3.1.9. Ký hiệu $M_{2\times 2}(\mathbb{R})$ là không gian các ma trận thực vuông cấp hai. Cho $W = \{ \begin{pmatrix} a & b \\ 0 & a+b+c \end{pmatrix} \text{ với } a, b, c \in \mathbb{R} \}$.}
\text{a) Chứng minh $W$ là một không gian con của $M_{2\times 2}(\mathbb{R})$.} \\
\text{b) Chứng minh $C = \{ \begin{pmatrix} 1 & 1 \\ 0 & 2 \end{pmatrix}, \begin{pmatrix} 0 & 1 \\ 0 & 1 \end{pmatrix}, \begin{pmatrix} 0 & 0 \\ 0 & 1 \end{pmatrix} \}$ là một cơ sở của không gian $W$.} \\
\text{c) Cho ánh xạ $f: W \to W, \begin{pmatrix} a & b \\ 0 & a+b+c \end{pmatrix} \mapsto \begin{pmatrix} a+c & b \\ 0 & a+b+c \end{pmatrix}$. Chứng minh $f$ là phép biến đổi tuyến tính trên $W$ và tìm ma trận của $f$ trong cơ sở $C$.} \\
\text{d) Tìm cơ sở của $\text{Im}f$ và xác định không gian $\text{Ker}f$.}

\section*{Giải.}

\text{a) Hiển nhiên $\mathbf{0} \in W$, nên } $W \ne \emptyset$. \text{Để chứng minh $W$ là một không gian con của $M_{2\times 2}(\mathbb{R})$, ta xét hai ma trận bất kỳ thuộc $W$: } $A = \begin{pmatrix} a & b \\ 0 & a+b+c \end{pmatrix}, B = \begin{pmatrix} a' & b' \\ 0 & a'+b'+c' \end{pmatrix} \text{ và } \alpha, \beta \in \mathbb{R}$ \text{ bất kỳ. Khi đó ta có}
$$
\alpha A + \beta B = \begin{pmatrix} \alpha a + \beta a' & \alpha b + \beta b' \\ 0 & \alpha(a+b+c) + \beta(a'+b'+c') \end{pmatrix}
$$
\text{Đặt } $a_{new} = \alpha a + \beta a', b_{new} = \alpha b + \beta b', c_{new} = \alpha c + \beta c'$. \text{Ta có}
$$
\alpha A + \beta B = \begin{pmatrix} a_{new} & b_{new} \\ 0 & a_{new} + b_{new} + c_{new} \end{pmatrix} \in W.
$$
\text{Vậy $W$ là một không gian con của $M_{2\times 2}(\mathbb{R})$.}

\text{b) Với mọi } $A = \begin{pmatrix} a & b \\ 0 & a+b+c \end{pmatrix} \in W$, \text{ ta có}
$$
A = a \begin{pmatrix} 1 & 0 \\ 0 & 1 \end{pmatrix} + b \begin{pmatrix} 0 & 1 \\ 0 & 1 \end{pmatrix} + c \begin{pmatrix} 0 & 0 \\ 0 & 1 \end{pmatrix}.
$$
\text{Vậy } $C' = \{ \begin{pmatrix} 1 & 0 \\ 0 & 1 \end{pmatrix}, \begin{pmatrix} 0 & 1 \\ 0 & 1 \end{pmatrix}, \begin{pmatrix} 0 & 0 \\ 0 & 1 \end{pmatrix} \}$ \text{ là cơ sở của } $W$ \text{ và } $\dim W = 3$. \\
\text{Để chứng minh } $C = \{ \begin{pmatrix} 1 & 1 \\ 0 & 2 \end{pmatrix}, \begin{pmatrix} 0 & 1 \\ 0 & 1 \end{pmatrix}, \begin{pmatrix} 0 & 0 \\ 0 & 1 \end{pmatrix} \}$ \text{ là một cơ sở của $W$, ta chứng minh hệ véctơ $C$ độc lập tuyến tính.}
\text{Xét } $\alpha \begin{pmatrix} 1 & 1 \\ 0 & 2 \end{pmatrix} + \beta \begin{pmatrix} 0 & 1 \\ 0 & 1 \end{pmatrix} + \gamma \begin{pmatrix} 0 & 0 \\ 0 & 1 \end{pmatrix} = \begin{pmatrix} 0 & 0 \\ 0 & 0 \end{pmatrix}$. \text{Ta có hệ phương trình}
$$
\begin{cases}
\alpha = 0 \\
\alpha + \beta = 0 \\
2\alpha + \beta + \gamma = 0
\end{cases}
\Leftrightarrow \alpha = \beta = \gamma = 0.
$$
\text{Vì } $\dim W = 3$ \text{ và hệ } $C$ \text{ có 3 véctơ độc lập tuyến tính, nên } $C$ \text{ là một cơ sở của không gian } $W$.

\text{c) Xét các ma trận } $A = \begin{pmatrix} a & b \\ 0 & a+b+c \end{pmatrix}, B = \begin{pmatrix} a' & b' \\ 0 & a'+b'+c' \end{pmatrix} \text{ thuộc } W \text{ và } \alpha, \beta \in \mathbb{R}$ \text{ bất kỳ. Ta có}
$$
f(\alpha A + \beta B) = f\left( \begin{pmatrix} \alpha a + \beta a' & \alpha b + \beta b' \\ 0 & \alpha(a+b+c) + \beta(a'+b'+c') \end{pmatrix} \right)
$$
$$
= \begin{pmatrix} (\alpha a + \beta a') + (\alpha c + \beta c') & \alpha b + \beta b' \\ 0 & \alpha(a+b+c) + \beta(a'+b'+c') \end{pmatrix}
$$
$$
= \alpha \begin{pmatrix} a+c & b \\ 0 & a+b+c \end{pmatrix} + \beta \begin{pmatrix} a'+c' & b' \\ 0 & a'+b'+c' \end{pmatrix}
$$
$$
= \alpha f(A) + \beta f(B).
$$
\text{Vậy $f$ là phép biến đổi tuyến tính trên $W$.}

\text{Ta tìm ma trận của $f$ trong cơ sở $C = \{C_1, C_2, C_3\}$ với $C_1 = \begin{pmatrix} 1 & 1 \\ 0 & 2 \end{pmatrix}, C_2 = \begin{pmatrix} 0 & 1 \\ 0 & 1 \end{pmatrix}, C_3 = \begin{pmatrix} 0 & 0 \\ 0 & 1 \end{pmatrix}$.}
\text{Ta tính ảnh của các véctơ cơ sở trong } $C$:
$$
f(C_1) = \begin{pmatrix} 1+0 & 1 \\ 0 & 1+1+0 \end{pmatrix} = \begin{pmatrix} 1 & 1 \\ 0 & 2 \end{pmatrix} = 1C_1 + 0C_2 + 0C_3
$$
$$
f(C_2) = \begin{pmatrix} 0+0 & 1 \\ 0 & 0+1+0 \end{pmatrix} = \begin{pmatrix} 0 & 1 \\ 0 & 1 \end{pmatrix} = 0C_1 + 1C_2 + 0C_3
$$
$$
f(C_3) = \begin{pmatrix} 0+1 & 0 \\ 0 & 0+0+1 \end{pmatrix} = \begin{pmatrix} 1 & 0 \\ 0 & 1 \end{pmatrix}.
$$
\text{Biểu diễn } $f(C_3)$ \text{ qua cơ sở } $C$: $f(C_3) = \alpha C_1 + \beta C_2 + \gamma C_3$.
$$
\begin{pmatrix} 1 & 0 \\ 0 & 1 \end{pmatrix} = \alpha \begin{pmatrix} 1 & 1 \\ 0 & 2 \end{pmatrix} + \beta \begin{pmatrix} 0 & 1 \\ 0 & 1 \end{pmatrix} + \gamma \begin{pmatrix} 0 & 0 \\ 0 & 1 \end{pmatrix}
$$
\text{Ta có hệ phương trình}
$$
\begin{cases}
\alpha = 1 \\
\alpha + \beta = 0 \\
2\alpha + \beta + \gamma = 1
\end{cases}
\Leftrightarrow
\begin{cases}
\alpha = 1 \\
\beta = -1 \\
\gamma = 0
\end{cases}
$$
\text{Vậy } $f(C_3) = 1C_1 - 1C_2 + 0C_3$.
\text{Ma trận của $f$ trong cơ sở $C$ là } $[f]_C = \begin{pmatrix} 1 & 0 & 1 \\ 0 & 1 & -1 \\ 0 & 0 & 0 \end{pmatrix}$.

\text{d) Ta tìm cơ sở của } $\text{Im}f$. \text{Vì } $\text{Im}f$ \text{ sinh bởi các cột của } $[f]_C$, \text{ và cột 1, 2 độc lập tuyến tính, nên } $\dim \text{Im}f = 2$.
\text{Cơ sở của } $\text{Im}f$ \text{ là } $\{C_1, C_2\} = \{ \begin{pmatrix} 1 & 1 \\ 0 & 2 \end{pmatrix}, \begin{pmatrix} 0 & 1 \\ 0 & 1 \end{pmatrix} \}$.

\text{Ta tìm không gian } $\text{Ker}f$. \text{Theo định lý số chiều } $\dim W = \dim \text{Ker}f + \dim \text{Im}f$, \text{ ta có } $3 = \dim \text{Ker}f + 2$, \text{ suy ra } $\dim \text{Ker}f = 1$.
\text{Xét } $X \in \text{Ker}f \Leftrightarrow [X]_C$ \text{ là nghiệm của hệ thuần nhất } $[f]_C [X]_C = \mathbf{0}$.
\text{Đặt } $[X]_C = \begin{pmatrix} \alpha \\ \beta \\ \gamma \end{pmatrix}$. \text{Ta có hệ phương trình}
$$
\begin{cases}
\alpha + \gamma = 0 \\
\beta - \gamma = 0
\end{cases}
\Leftrightarrow
\begin{cases}
\alpha = -C \\
\beta = C \\
\gamma = C
\end{cases}
$$
\text{Chọn } $C=1$, \text{ ta có } $[X]_C = \begin{pmatrix} -1 \\ 1 \\ 1 \end{pmatrix}$. \text{Ma trận } $X$ \text{ tương ứng là}
$$
X = -1C_1 + 1C_2 + 1C_3 = -1 \begin{pmatrix} 1 & 1 \\ 0 & 2 \end{pmatrix} + 1 \begin{pmatrix} 0 & 1 \\ 0 & 1 \end{pmatrix} + 1 \begin{pmatrix} 0 & 0 \\ 0 & 1 \end{pmatrix} = \begin{pmatrix} -1 & 0 \\ 0 & 0 \end{pmatrix}.
$$
\text{Vậy } $\text{Ker}f = \mathcal{L}\{ \begin{pmatrix} -1 & 0 \\ 0 & 0 \end{pmatrix} \}$.

\end{document}
