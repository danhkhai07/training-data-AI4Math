\documentclass{article}
\usepackage[utf8]{inputenc}
\usepackage{amsmath, amssymb}
\usepackage{cases}

\begin{document}

\subsection*{Bài 3.1.2. Cho $f: P_2[x] \to P_2[x], p(x) \mapsto (x+1)p'(x)$.}
\text{a) Chứng minh $f$ là một phép biến đổi tuyến tính.} \\
\text{b) Tìm ma trận của $f$ trong cơ sở chính tắc của $P_2[x]$.}

\section*{Giải.}

\text{a) Để chứng minh $f$ là phép biến đổi tuyến tính, ta kiểm tra đẳng thức } $f(\alpha p(x) + \beta q(x)) = \alpha f(p(x)) + \beta f(q(x))$.
\text{Thật vậy, ta có}
\begin{align*}
f(\alpha p(x) + \beta q(x)) &= (x+1)(\alpha p(x) + \beta q(x))' \\
&= (x+1)(\alpha p'(x) + \beta q'(x)) \\
&= (x+1)\alpha p'(x) + (x+1)\beta q'(x) \\
&= \alpha [(x+1) p'(x)] + \beta [(x+1) q'(x)] \\
&= \alpha f(p(x)) + \beta f(q(x)).
\end{align*}
\text{Vậy $f$ là một phép biến đổi tuyến tính.}

\text{b) Gọi } $E = \{1, x, x^2\}$ \text{ là cơ sở chính tắc của } $P_2[x]$. \text{Để tìm ma trận của } $f$ \text{ trong cơ sở } $E$, \text{ ta tìm ảnh của các véctơ cơ sở } $E$ \text{ và biểu diễn chúng trong cơ sở } $E$ \text{ như sau:}
$$
f(1) = (x+1)(1)' = (x+1) \cdot 0 = 0 = 0 \cdot 1 + 0 \cdot x + 0 \cdot x^2
$$
$$
f(x) = (x+1)(x)' = (x+1) \cdot 1 = 1 + x = 1 \cdot 1 + 1 \cdot x + 0 \cdot x^2
$$
$$
f(x^2) = (x+1)(x^2)' = (x+1) \cdot 2x = 2x^2 + 2x = 0 \cdot 1 + 2 \cdot x + 2 \cdot x^2
$$
\text{Vậy ma trận của } $f$ \text{ trong cơ sở } $E$ \text{ là } $A = \begin{pmatrix} 0 & 1 & 0 \\ 0 & 1 & 2 \\ 0 & 0 & 2 \end{pmatrix}$.

\end{document}
