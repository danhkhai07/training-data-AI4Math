\documentclass{article}
\usepackage[utf8]{inputenc}
\usepackage{amsmath, amssymb}

\begin{document}

\subsection*{Xét tính độc lập tuyến tính của các vectơ sau:}
\begin{enumerate}
    \item[a)] $\mathbf{u}_1 = (1, 2, 3), \mathbf{u}_2 = (3, 2, 1), \mathbf{u}_3 = (1, 1, 1)$ trong $\mathbb{R}^3$.
    \item[b)] $\mathbf{u}_1 = (1, 2, 3, 4), \mathbf{u}_2 = (2, 3, 4, 5), \mathbf{u}_3 = (3, 4, 5, 6), \mathbf{u}_4 = (4, 5, 6, 7)$ trong $\mathbb{R}^4$.
    \item[c)] $p_1(x) = 1, p_2(x) = x, p_3(x) = x^2$ trong $P_2[x]$.
\end{enumerate}

\section*{Giải.}

\subsection*{a) Xét các vectơ $\mathbf{u}_1, \mathbf{u}_2, \mathbf{u}_3$ trong $\mathbb{R}^3$}
\text{Các vectơ độc lập tuyến tính khi và chỉ khi $\det A \ne 0$, với $A$ là ma trận lập bởi các vectơ cột (hoặc hàng).}
$$
A = \begin{pmatrix} 1 & 3 & 1 \\ 2 & 2 & 1 \\ 3 & 1 & 1 \end{pmatrix}
$$
\text{Tính định thức $\det A$ (khai triển theo cột 3):}
$$
\det A = 1 \begin{vmatrix} 2 & 2 \\ 3 & 1 \end{vmatrix} - 1 \begin{vmatrix} 1 & 3 \\ 3 & 1 \end{vmatrix} + 1 \begin{vmatrix} 1 & 3 \\ 2 & 2 \end{vmatrix}
$$
$$
= 1(2 - 6) - 1(1 - 9) + 1(2 - 6) = -4 - (-8) + (-4) = -4 + 8 - 4 = 0.
$$
\text{Vì $\det A = 0$, ba vectơ $\mathbf{u}_1, \mathbf{u}_2, \mathbf{u}_3$ **phụ thuộc tuyến tính**.}

\subsection*{b) Xét các vectơ $\mathbf{u}_1, \mathbf{u}_2, \mathbf{u}_3, \mathbf{u}_4$ trong $\mathbb{R}^4$}
\text{Ta xét phương trình tổ hợp tuyến tính thuần nhất:}
$$
\alpha_1 \mathbf{u}_1 + \alpha_2 \mathbf{u}_2 + \alpha_3 \mathbf{u}_3 + \alpha_4 \mathbf{u}_4 = \mathbf{0}
$$
\text{Ta xét ma trận lập bởi các vectơ (viết theo hàng để dễ biến đổi Gauss):}
$$
A = \begin{pmatrix}
1 & 2 & 3 & 4 \\
2 & 3 & 4 & 5 \\
3 & 4 & 5 & 6 \\
4 & 5 & 6 & 7
\end{pmatrix}
$$
\text{Thực hiện $\text{H}_2 \leftarrow \text{H}_2 - 2\text{H}_1$, $\text{H}_3 \leftarrow \text{H}_3 - 3\text{H}_1$, $\text{H}_4 \leftarrow \text{H}_4 - 4\text{H}_1$:}
$$
\xrightarrow[\text{H}_4 \leftarrow \text{H}_4 - 4\text{H}_1]{\text{H}_2 \leftarrow \text{H}_2 - 2\text{H}_1, \text{H}_3 \leftarrow \text{H}_3 - 3\text{H}_1}
\begin{pmatrix}
1 & 2 & 3 & 4 \\
0 & -1 & -2 & -3 \\
0 & -2 & -4 & -6 \\
0 & -3 & -6 & -9
\end{pmatrix}
$$
\text{Thực hiện $\text{H}_3 \leftarrow \text{H}_3 - 2\text{H}_2$ và $\text{H}_4 \leftarrow \text{H}_4 - 3\text{H}_2$:}
$$
\xrightarrow[\text{H}_4 \leftarrow \text{H}_4 - 3\text{H}_2]{\text{H}_3 \leftarrow \text{H}_3 - 2\text{H}_2}
\begin{pmatrix}
1 & 2 & 3 & 4 \\
0 & -1 & -2 & -3 \\
0 & 0 & 0 & 0 \\
0 & 0 & 0 & 0
\end{pmatrix}
$$
\text{Hạng của ma trận $r(A) = 2 < 4$ (số vectơ). Hệ thuần nhất có nghiệm không tầm thường.}
\text{Vậy bốn vectơ **phụ thuộc tuyến tính**.}

\subsection*{c) Xét các đa thức $p_1(x) = 1, p_2(x) = x, p_3(x) = x^2$ trong $P_2[x]$}
\text{Ta xét phương trình tổ hợp tuyến tính thuần nhất:}
$$
\alpha_1 p_1(x) + \alpha_2 p_2(x) + \alpha_3 p_3(x) = \mathbf{0}(x)
$$
$$
\alpha_1 (1) + \alpha_2 (x) + \alpha_3 (x^2) = 0, \quad \forall x \in \mathbb{R}.
$$
\text{Đẳng thức này là một đa thức zero. Theo định nghĩa đa thức zero, tất cả các hệ số phải bằng 0.}
$$
\alpha_1 = 0, \quad \alpha_2 = 0, \quad \alpha_3 = 0.
$$
\text{Phương trình tổ hợp tuyến tính chỉ có nghiệm tầm thường.}
\text{Vậy ba đa thức $p_1(x), p_2(x), p_3(x)$ **độc lập tuyến tính**.}

\end{document}
