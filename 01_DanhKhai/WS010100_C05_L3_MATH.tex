\documentclass{article}
\usepackage[utf8]{inputenc}
\usepackage{amsmath, amssymb}

\begin{document}

\subsection*{Bài 3.1.30. Cho $f$ là một phép biến đổi tuyến tính trên $\mathbb{R}^3$ có ma trận trong cơ sở chính tắc là}
$$
A = \begin{pmatrix} 2 & -1 & 1 \\ -1 & 2 & -1 \\ 0 & 0 & 1 \end{pmatrix}.
$$
\text{Tìm một cơ sở của không gian $\mathbb{R}^3$ sao cho ma trận của $f$ trong cơ sở đó là ma trận chéo.}

\section*{Giải.}
\text{Đa thức đặc trưng của ma trận $A$ là}
$$
\det(A - \lambda I) = \det \begin{pmatrix} 2 - \lambda & -1 & 1 \\ -1 & 2 - \lambda & -1 \\ 0 & 0 & 1 - \lambda \end{pmatrix} = (1 - \lambda)^2 (3 - \lambda).
$$
\text{Do đó $\det(A - \lambda I) = 0$ khi và chỉ khi $\lambda = 3$ hoặc $\lambda = 1$. Vậy $f$ có các giá trị riêng $\lambda_1 = 3$ và $\lambda_2 = 1$.}

\begin{itemize}
    \item \text{Tọa độ vectơ riêng của $f$ ứng với giá trị riêng $\lambda_1 = 3$ là nghiệm không tầm thường của hệ phương trình $(A - 3I)x = 0$, tức là}
    $$
    \begin{pmatrix} -1 & -1 & 1 \\ -1 & -1 & -1 \\ 0 & 0 & -2 \end{pmatrix} \begin{pmatrix} x_1 \\ x_2 \\ x_3 \end{pmatrix} = \begin{pmatrix} 0 \\ 0 \\ 0 \end{pmatrix}.
    $$
\end{itemize}

\text{Xét ma trận các hệ số mở rộng}
$$
\begin{pmatrix}
-1 & -1 & 1 & | & 0 \\
-1 & -1 & -1 & | & 0 \\
0 & 0 & -2 & | & 0
\end{pmatrix}
\xrightarrow{H_2 \to H_2 - H_1}
\begin{pmatrix}
-1 & -1 & 1 & | & 0 \\
0 & 0 & -2 & | & 0 \\
0 & 0 & -2 & | & 0
\end{pmatrix}
\xrightarrow{H_3 \to H_3 - H_2}
\begin{pmatrix}
-1 & -1 & 1 & | & 0 \\
0 & 0 & -2 & | & 0 \\
0 & 0 & 0 & | & 0
\end{pmatrix}.
$$
\text{Ta nhận được phương trình tương đương}
$$
\begin{cases}
-x_1 - x_2 + x_3 = 0 \\
-2x_3 = 0
\end{cases}
\Leftrightarrow
\begin{cases}
x_3 = 0 \\
x_1 = -x_2
\end{cases}
$$
\text{Từ đó suy ra $(x_1, x_2, x_3) = C(1, -1, 0)$ với $C$ là hằng số.}
\text{Do $A$ là ma trận của $f$ trong cơ sở chính tắc của $\mathbb{R}^3$ nên ta chọn một vectơ riêng ứng với giá trị riêng $\lambda_1 = 3$ là $u_1 = (1, -1, 0)$.}

\begin{itemize}
    \item \text{Tọa độ vectơ riêng của $f$ ứng với giá trị riêng $\lambda_2 = 1$ là nghiệm không tầm thường của hệ phương trình $(A - I)x = 0$, tức là}
    $$
    \begin{pmatrix} 1 & -1 & 1 \\ -1 & 1 & -1 \\ 0 & 0 & 0 \end{pmatrix} \begin{pmatrix} x_1 \\ x_2 \\ x_3 \end{pmatrix} = \begin{pmatrix} 0 \\ 0 \\ 0 \end{pmatrix}.
    $$
\end{itemize}

\text{Xét ma trận các hệ số mở rộng}
$$
\begin{pmatrix}
1 & -1 & 1 & | & 0 \\
-1 & 1 & -1 & | & 0 \\
0 & 0 & 0 & | & 0
\end{pmatrix}
\xrightarrow{H_2 \to H_2 + H_1}
\begin{pmatrix}
1 & -1 & 1 & | & 0 \\
0 & 0 & 0 & | & 0 \\
0 & 0 & 0 & | & 0
\end{pmatrix}.
$$
\text{Ta nhận được phương trình tương đương}
$$
x_1 - x_2 + x_3 = 0 \Leftrightarrow x_1 = x_2 - x_3.
$$
\text{Từ đó suy ra $(x_1, x_2, x_3) = (x_2 - x_3, x_2, x_3) = x_2(1, 1, 0) + x_3(-1, 0, 1) = C_1(1, 1, 0) + C_2(-1, 0, 1)$ với $C_1, C_2$ là hằng số.}
\text{Do $A$ là ma trận của $f$ trong cơ sở chính tắc của $\mathbb{R}^3$ nên ta chọn hai vectơ riêng độc lập tuyến tính của $f$ ứng với giá trị riêng $\lambda_2 = 1$ là}
$$
u_2 = (1, 1, 0) \quad \text{và} \quad u_3 = (0, 1, 1).
$$
\text{Ta có hệ gồm ba vectơ riêng}
$$
S = \{u_1 = (1, -1, 0), u_2 = (1, 1, 0), u_3 = (0, 1, 1)\}
$$
\text{độc lập tuyến tính trong $\mathbb{R}^3$ nên chúng lập thành cơ sở. Vậy trong cơ sở $\{u_1, u_2, u_3\}$ thì ma trận của phép biến đổi tuyến tính $f$ có dạng chéo}
$$
D = \begin{pmatrix}
3 & 0 & 0 \\
0 & 1 & 0 \\
0 & 0 & 1
\end{pmatrix}.
$$

\end{document}
