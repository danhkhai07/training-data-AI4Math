\documentclass{article}
\usepackage[utf8]{inputenc}
\usepackage{amsmath, amssymb}

\begin{document}

\subsection*{Trong không gian với hệ tọa độ Đề-các $Oxyz$ cho các mặt phẳng}
$$
\begin{cases}
x - 2y + z = mx \\
3x - y - 2z = my \\
3x - 2y - z = mz
\end{cases}
$$
\text{Xác định giá trị của $m$ để ba mặt phẳng đó chứa cùng một đường thẳng.}

\section*{Giải.}

\text{Điều kiện để ba mặt phẳng (không trùng nhau) chứa cùng một đường thẳng là hệ phương trình tương ứng có vô số nghiệm.}

\text{Ta biến đổi hệ phương trình đã cho về dạng thuần nhất chuẩn:}
$$
\begin{cases}
(1-m)x - 2y + z = 0 \\
3x + (-1-m)y - 2z = 0 \\
3x - 2y + (-1-m)z = 0
\end{cases}
$$

\text{Hệ phương trình thuần nhất có vô số nghiệm khi và chỉ khi định thức của ma trận hệ số bằng không ($\det A = 0$).}
$$
\det A = \begin{vmatrix} 1-m & -2 & 1 \\ 3 & -1-m & -2 \\ 3 & -2 & -1-m \end{vmatrix}
$$

\text{Thực hiện phép biến đổi $\text{H}_2 \leftarrow \text{H}_2 - \text{H}_3$:}
$$
\det A = \begin{vmatrix} 1-m & -2 & 1 \\ 0 & -1-m - (-2) & -2 - (-1-m) \\ 3 & -2 & -1-m \end{vmatrix}
$$
$$
= \begin{vmatrix} 1-m & -2 & 1 \\ 0 & 1-m & m-1 \\ 3 & -2 & -1-m \end{vmatrix}
$$
\text{Đưa nhân tử chung $(1-m)$ ra khỏi hàng 2:}
$$
\det A = (1-m) \begin{vmatrix} 1-m & -2 & 1 \\ 0 & 1 & -1 \\ 3 & -2 & -1-m \end{vmatrix}
$$
\text{Thực hiện $\text{C}_2 \leftarrow \text{C}_2 + 2\text{C}_3$:}
$$
\det A = (1-m) \begin{vmatrix} 1-m & -2+2 & 1 \\ 0 & 1-2 & -1 \\ 3 & -2-2(1+m) & -1-m \end{vmatrix}
$$
\text{Thực hiện $\text{C}_2 \leftarrow \text{C}_2 + \text{C}_3$ (hoặc $\text{C}_3 \leftarrow \text{C}_3 + \text{C}_2$ cho gọn):}
\text{Áp dụng $\text{C}_1 \leftarrow \text{C}_1 + \text{C}_2 + \text{C}_3$ (giống cách giải của bài toán tương tự):}
$$
\det A = \begin{vmatrix} 1-m + (-2) + 1 & -2 & 1 \\ 3 + (-1-m) + (-2) & -1-m & -2 \\ 3 + (-2) + (-1-m) & -2 & -1-m \end{vmatrix}
$$
$$
= \begin{vmatrix} -m & -2 & 1 \\ -m & -1-m & -2 \\ -m & -2 & -1-m \end{vmatrix} = -m \begin{vmatrix} 1 & -2 & 1 \\ 1 & -1-m & -2 \\ 1 & -2 & -1-m \end{vmatrix}
$$
\text{Thực hiện $\text{H}_2 \leftarrow \text{H}_2 - \text{H}_1$ và $\text{H}_3 \leftarrow \text{H}_3 - \text{H}_1$:}
$$
\det A = -m \begin{vmatrix} 1 & -2 & 1 \\ 0 & 1-m & -3 \\ 0 & 0 & -2-m \end{vmatrix}
$$
\text{Đây là định thức của ma trận tam giác:}
$$
\det A = -m \cdot 1 \cdot (1-m) \cdot (-2-m) = m(1-m)(2+m).
$$

\text{Để hệ có vô số nghiệm, $\det A = 0$, tức là:}
$$
m(1-m)(2+m) = 0 \Leftrightarrow m = 0, m = 1, \text{ hoặc } m = -2.
$$

\text{Điều kiện để ba mặt phẳng chứa cùng một đường thẳng là hệ phương trình có vô số nghiệm.}
\text{Giá trị của $m$ là $m=1, m=0, \text{ hoặc } m=-2$.}

\end{document}
