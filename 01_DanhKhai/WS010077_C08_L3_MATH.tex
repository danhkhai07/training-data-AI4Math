\documentclass{article}
\usepackage[utf8]{inputenc}
\usepackage{amsmath, amssymb}
\usepackage{cases}

\begin{document}

\subsection*{Bài 3.1.3. Cho $f: M_{2\times 2}(\mathbb{R}) \to M_{2\times 2}(\mathbb{R}), X \mapsto AX$, trong đó $A = \begin{pmatrix} -2 & 1 \\ 4 & 6 \end{pmatrix}$.}
\text{a) Chứng minh $f$ là phép biến đổi tuyến tính.} \\
\text{b) Tìm ma trận của $f$ trong cơ sở chính tắc $E$ của $M_{2\times 2}(\mathbb{R})$.}

\section*{Giải.}

\text{a) Ta chứng minh đẳng thức } $f(\alpha X + \beta Y) = \alpha f(X) + \beta f(Y)$.
\text{Thật vậy, với } $X, Y \in M_{2\times 2}(\mathbb{R})$ \text{ và } $\alpha, \beta \in \mathbb{R}$ \text{ tùy ý, ta có}
\begin{align*}
f(\alpha X + \beta Y) &= A(\alpha X + \beta Y) \\
&= \alpha AX + \beta AY \\
&= \alpha f(X) + \beta f(Y).
\end{align*}
\text{Vậy $f$ là phép biến đổi tuyến tính.}

\text{b) Gọi cơ sở chính tắc của } $M_{2\times 2}(\mathbb{R})$ \text{ là }
$$
E = \{E_1 = \begin{pmatrix} 1 & 0 \\ 0 & 0 \end{pmatrix}, E_2 = \begin{pmatrix} 0 & 1 \\ 0 & 0 \end{pmatrix}, E_3 = \begin{pmatrix} 0 & 0 \\ 1 & 0 \end{pmatrix}, E_4 = \begin{pmatrix} 0 & 0 \\ 0 & 1 \end{pmatrix}\}.
$$
\text{Để tìm ma trận của } $f$, \text{ ta tìm ảnh của các véctơ trong cơ sở } $E$ \text{ và tìm tọa độ của các ảnh đó trong cơ sở } $E$ \text{ như sau:}
$$
f(E_1) = \begin{pmatrix} -2 & 1 \\ 4 & 6 \end{pmatrix} \begin{pmatrix} 1 & 0 \\ 0 & 0 \end{pmatrix} = \begin{pmatrix} -2 & 0 \\ 4 & 0 \end{pmatrix} = -2E_1 + 4E_3
$$
$$
f(E_2) = \begin{pmatrix} -2 & 1 \\ 4 & 6 \end{pmatrix} \begin{pmatrix} 0 & 1 \\ 0 & 0 \end{pmatrix} = \begin{pmatrix} 0 & -2 \\ 0 & 4 \end{pmatrix} = -2E_2 + 4E_4
$$
$$
f(E_3) = \begin{pmatrix} -2 & 1 \\ 4 & 6 \end{pmatrix} \begin{pmatrix} 0 & 0 \\ 1 & 0 \end{pmatrix} = \begin{pmatrix} 1 & 0 \\ 6 & 0 \end{pmatrix} = 1E_1 + 6E_3
$$
$$
f(E_4) = \begin{pmatrix} -2 & 1 \\ 4 & 6 \end{pmatrix} \begin{pmatrix} 0 & 0 \\ 0 & 1 \end{pmatrix} = \begin{pmatrix} 0 & 1 \\ 0 & 6 \end{pmatrix} = 1E_2 + 6E_4
$$
\text{Vậy ma trận của } $f$ \text{ trong cơ sở } $E$ \text{ là } $A = \begin{pmatrix} -2 & 0 & 1 & 0 \\ 0 & -2 & 0 & 1 \\ 4 & 0 & 6 & 0 \\ 0 & 4 & 0 & 6 \end{pmatrix}$.

\end{document}
