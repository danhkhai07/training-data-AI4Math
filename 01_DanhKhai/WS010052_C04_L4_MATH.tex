\documentclass{article}
\usepackage[utf8]{inputenc}
\usepackage{amsmath, amssymb}

\begin{document}

\subsection*{Trong không gian $P_3[x]$ cho $h = x^3 - 2x^2 + x + 1$. Chứng minh rằng $B = \{h, h', h'', h''' \}$ là một cơ sở của $P_3[x]$. Tìm ma trận chuyển cơ sở từ cơ sở chính tắc sang cơ sở $B$. Hãy tìm tọa độ của $q(x) = 10x^3 - 8x^2 + ax + b$ trong cơ sở $B$.}

\section*{Giải.}

\text{Ta có}
$$
h(x) = x^3 - 2x^2 + x + 1
$$
$$
h'(x) = 3x^2 - 4x + 1
$$
$$
h''(x) = 6x - 4
$$
$$
h'''(x) = 6
$$
\text{Ta biểu diễn $h, h', h'', h'''$ qua cơ sở chính tắc $E = \{1, x, x^2, x^3\}$:}
$$
h(x) = 1 \cdot 1 + 1 \cdot x + (-2) \cdot x^2 + 1 \cdot x^3
$$
$$
h'(x) = 1 \cdot 1 + (-4) \cdot x + 3 \cdot x^2 + 0 \cdot x^3
$$
$$
h''(x) = (-4) \cdot 1 + 6 \cdot x + 0 \cdot x^2 + 0 \cdot x^3
$$
$$
h'''(x) = 6 \cdot 1 + 0 \cdot x + 0 \cdot x^2 + 0 \cdot x^3
$$

\text{Suy ra ma trận chuyển từ cơ sở chính tắc $E = \{1, x, x^2, x^3\}$ sang cơ sở $B$ là (theo ký hiệu trong hình ảnh $T$ là ma trận $T_{B \to E}$):}
$$
T = \begin{pmatrix}
1 & 1 & -4 & 6 \\
1 & -4 & 6 & 0 \\
-2 & 3 & 0 & 0 \\
1 & 0 & 0 & 0
\end{pmatrix}.
$$

\text{Để tìm tọa độ của $q(x) = 10x^3 - 8x^2 + ax + b$ trong cơ sở $B$, ta giải phương trình $q = \alpha_1 h + \alpha_2 h' + \alpha_3 h'' + \alpha_4 h'''$ hay}
$$
b + ax - 8x^2 + 10x^3 = (\alpha_1 + \alpha_2 - 4\alpha_3 + 6\alpha_4) + (\alpha_1 - 4\alpha_2 + 6\alpha_3)x + (-2\alpha_1 + 3\alpha_2)x^2 + \alpha_1 x^3.
$$

\text{Ta thu được hệ phương trình}
$$
\begin{cases}
\alpha_1 + \alpha_2 - 4\alpha_3 + 6\alpha_4 = b \\
\alpha_1 - 4\alpha_2 + 6\alpha_3 = a \\
-2\alpha_1 + 3\alpha_2 = -8 \\
\alpha_1 = 10
\end{cases}
\Leftrightarrow
\begin{cases}
\alpha_1 = 10 \\
\alpha_2 = \frac{10}{3} \\
\alpha_3 = \frac{a}{6} + 1 \\
\alpha_4 = \frac{b}{6} + \frac{1}{6} + \frac{5}{3}.
\end{cases}
$$
\text{Vậy tọa độ của $q(x)$ trong cơ sở $B$ là $\left( 10, \frac{4}{3}, \frac{a}{6} + 1, \frac{b}{6} + \frac{5}{3} \right)$.}

\end{document}
