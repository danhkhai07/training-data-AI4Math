\documentclass{article}
\usepackage[utf8]{inputenc}
\usepackage{amsmath, amssymb}

\begin{document}

\subsection*{Bài 2.1.20. Tìm một cơ sở và tìm chiều của không gian}
$$
V = \left\{ A = \begin{pmatrix} a & b \\ c & d \end{pmatrix} \in M_{2\times 2}(\mathbb{R}) \middle| a = d, b = -c \right\}.
$$

\section*{Giải.}

\text{Với} $A = \begin{pmatrix} a & b \\ c & d \end{pmatrix} \in V$, \text{ta có}
$$
A = \begin{pmatrix} a & b \\ c & d \end{pmatrix} \underset{a=d, b=-c}{=} \begin{pmatrix} a & b \\ -b & a \end{pmatrix} = a \cdot \begin{pmatrix} 1 & 0 \\ 0 & 1 \end{pmatrix} + b \cdot \begin{pmatrix} 0 & 1 \\ -1 & 0 \end{pmatrix}.
$$
\text{Suy ra hệ}
$$
\left\{ u_1 = \begin{pmatrix} 1 & 0 \\ 0 & 1 \end{pmatrix}; u_2 = \begin{pmatrix} 0 & 1 \\ -1 & 0 \end{pmatrix} \right\} \text{ là một hệ sinh của } V.
$$
\text{Dễ dàng chứng minh được} $\{u_1; u_2\}$ \text{độc lập tuyến tính.}

\text{Vậy} $\{u_1; u_2\}$ \text{là một cơ sở của không gian} $V$ \text{và} $\dim V = 2$.

\end{document}
