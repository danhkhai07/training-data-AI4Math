\documentclass{article}
\usepackage[utf8]{inputenc}
\usepackage{amsmath, amssymb}
\usepackage{cases}

\begin{document}

\subsection*{Bài 3.1.14. Cho ánh xạ tuyến tính $f: \mathbb{R}^3 \to \mathbb{R}^3$. Biết $f(1, 1, 1) = (2, -1, 3), f(0, 1, 1) = (0, -1, 2), f(0, 0, 1) = (0, 1, -1)$.}
\text{a) Tìm ma trận của $f$ trong cơ sở chính tắc $E = \{\mathbf{e}_1, \mathbf{e}_2, \mathbf{e}_3\}$ của $\mathbb{R}^3$.} \\
\text{b) Chứng minh $F = \{\mathbf{f}_1 = (1, 1, 1), \mathbf{f}_2 = (0, 1, 1), \mathbf{f}_3 = (0, 0, 1)\}$ là một cơ sở của $\mathbb{R}^3$.} \\
\text{c) Tìm ma trận của $f$ trong cơ sở $F$.}

\section*{Giải.}

\text{b) Ta chứng minh hệ véctơ $F = \{\mathbf{f}_1, \mathbf{f}_2, \mathbf{f}_3\}$ độc lập tuyến tính.}
\text{Xét ma trận chứa các véctơ $F$ làm cột:}
$$
A_F = \begin{pmatrix}
1 & 0 & 0 \\
1 & 1 & 0 \\
1 & 1 & 1
\end{pmatrix}.
$$
\text{Vì } $\det(A_F) = 1 \cdot 1 \cdot 1 = 1 \ne 0$, \text{ nên } $F$ \text{ là hệ véctơ độc lập tuyến tính. Do } $\dim \mathbb{R}^3 = 3$ \text{ nên } $F$ \text{ là một cơ sở của } $\mathbb{R}^3$.

\text{a) Ta tìm ma trận của $f$ trong cơ sở chính tắc $E$ là } $A = [f]_E = \begin{pmatrix} [f(\mathbf{e}_1)]_E & [f(\mathbf{e}_2)]_E & [f(\mathbf{e}_3)]_E \end{pmatrix}$.
\text{Các véctơ cơ sở chính tắc } $\mathbf{e}_i$ \text{ được biểu diễn qua cơ sở } $F$:
$$
\mathbf{e}_3 = (0, 0, 1) = \mathbf{f}_3 \implies f(\mathbf{e}_3) = f(\mathbf{f}_3) = (0, 1, -1).
$$
$$
\mathbf{e}_2 = (0, 1, 0) = \mathbf{f}_2 - \mathbf{f}_3 \implies f(\mathbf{e}_2) = f(\mathbf{f}_2) - f(\mathbf{f}_3) = (0, -1, 2) - (0, 1, -1) = (0, -2, 3).
$$
$$
\mathbf{e}_1 = (1, 0, 0) = \mathbf{f}_1 - \mathbf{f}_2 \implies f(\mathbf{e}_1) = f(\mathbf{f}_1) - f(\mathbf{f}_2) = (2, -1, 3) - (0, -1, 2) = (2, 0, 1).
$$
\text{Ma trận của $f$ trong cơ sở chính tắc $E$ là}
$$
A = [f]_E = \begin{pmatrix}
2 & 0 & 0 \\
0 & -2 & 1 \\
1 & 3 & -1
\end{pmatrix}.
$$

\text{c) Ma trận của $f$ trong cơ sở $F = \{\mathbf{f}_1, \mathbf{f}_2, \mathbf{f}_3\}$ là } $[f]_F = \begin{pmatrix} [f(\mathbf{f}_1)]_F & [f(\mathbf{f}_2)]_F & [f(\mathbf{f}_3)]_F \end{pmatrix}$.
\text{Ta có tọa độ của các véctơ ảnh $f(\mathbf{f}_i)$ trong cơ sở $E$: } $f(\mathbf{f}_1) = (2, -1, 3), f(\mathbf{f}_2) = (0, -1, 2), f(\mathbf{f}_3) = (0, 1, -1)$.
\text{Gọi } $[f(\mathbf{f}_i)]_F = \begin{pmatrix} a \\ b \\ c \end{pmatrix}. \text{Ta có } f(\mathbf{f}_i) = a\mathbf{f}_1 + b\mathbf{f}_2 + c\mathbf{f}_3$.

\text{Với } $f(\mathbf{f}_1) = (2, -1, 3)$:
$$
\begin{cases}
a = 2 \\
a + b = -1 \\
a + b + c = 3
\end{cases}
\Leftrightarrow
\begin{cases}
a = 2 \\
b = -3 \\
c = 4
\end{cases}
$$
\text{Với } $f(\mathbf{f}_2) = (0, -1, 2)$:
$$
\begin{cases}
a = 0 \\
a + b = -1 \\
a + b + c = 2
\end{cases}
\Leftrightarrow
\begin{cases}
a = 0 \\
b = -1 \\
c = 3
\end{cases}
$$
\text{Với } $f(\mathbf{f}_3) = (0, 1, -1)$:
$$
\begin{cases}
a = 0 \\
a + b = 1 \\
a + b + c = -1
\end{cases}
\Leftrightarrow
\begin{cases}
a = 0 \\
b = 1 \\
c = -2
\end{cases}
$$
\text{Ma trận của $f$ trong cơ sở $F$ là}
$$
[f]_F = \begin{pmatrix}
2 & 0 & 0 \\
-3 & -1 & 1 \\
4 & 3 & -2
\end{pmatrix}.
$$

\end{document}
