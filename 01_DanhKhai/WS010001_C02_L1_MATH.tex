\documentclass{article}
\usepackage[utf8]{inputenc}
\usepackage{amsmath, amssymb}

\begin{document}

\subsection*{Bài 1.1.1. Thực hiện phép nhân các ma trận sau:}
\begin{enumerate}
    \item[a)]
    $$
    \begin{pmatrix}
    1 & 3 & 2 \\
    -4 & -1 & 5
    \end{pmatrix}
    \begin{pmatrix}
    0 & 3 \\
    -2 & -4 \\
    1 & 5
    \end{pmatrix}
    $$
    \item[b)]
    $$
    \begin{pmatrix}
    -1 & 2 & 3 \\
    2 & 1 & 4
    \end{pmatrix}
    \begin{pmatrix}
    2 & -1 \\
    3 & 1 \\
    1 & 5
    \end{pmatrix}
    \begin{pmatrix}
    1 & 4 \\
    2 & 3
    \end{pmatrix}
    $$
    \item[c)]
    $$
    \begin{pmatrix}
    \cos\theta & -\sin\theta \\
    \sin\theta & \cos\theta
    \end{pmatrix}^n, \quad n \in \mathbb{N}, n \ge 2.
    $$
\end{enumerate}

\section*{Giải.}

\subsection*{a) Ta có}
$$
\begin{pmatrix}
1 & 3 & 2 \\
-4 & -1 & 5
\end{pmatrix}
\begin{pmatrix}
0 & 3 \\
-2 & -4 \\
1 & 5
\end{pmatrix}
= \begin{pmatrix}
1\cdot 0 + 3\cdot (-2) + 2\cdot 1 & 1\cdot 3 + 3\cdot (-4) + 2\cdot 5 \\
-4\cdot 0 + (-1)\cdot (-2) + 5\cdot 1 & -4\cdot 3 + (-1)\cdot (-4) + 5\cdot 5
\end{pmatrix}
= \begin{pmatrix}
-4 & 1 \\
7 & 17
\end{pmatrix}
$$

\subsection*{b) Tương tự ta có}
$$
\begin{pmatrix}
-1 & 2 & 3 \\
2 & 1 & 4
\end{pmatrix}
\begin{pmatrix}
2 & -1 \\
3 & 1 \\
1 & 5
\end{pmatrix}
= \begin{pmatrix}
7 & 18 \\
3 & 21
\end{pmatrix}
$$
$$
\begin{pmatrix}
7 & 18 \\
3 & 21
\end{pmatrix}
\begin{pmatrix}
1 & 4 \\
2 & 3
\end{pmatrix}
= \begin{pmatrix}
7\cdot 1 + 18\cdot 2 & 7\cdot 4 + 18\cdot 3 \\
3\cdot 1 + 21\cdot 2 & 3\cdot 4 + 21\cdot 3
\end{pmatrix}
= \begin{pmatrix}
43 & 82 \\
45 & 75
\end{pmatrix}
$$
\text{(Lưu ý: Kết quả cuối cùng trong hình ảnh là $\begin{pmatrix} 43 & 82 \\ -39 & 51 \end{pmatrix}$. Tôi đã chép lại các bước tính toán và ghi chú kết quả trong hình. Dựa trên phép tính: $3\cdot 1 + 21\cdot 2 = 3 + 42 = 45$, và $3\cdot 4 + 21\cdot 3 = 12 + 63 = 75$, kết quả chính xác phải là $\begin{pmatrix} 43 & 82 \\ 45 & 75 \end{pmatrix}$.)}

\subsection*{c) Đặt $A = \begin{pmatrix} \cos\theta & -\sin\theta \\ \sin\theta & \cos\theta \end{pmatrix}$. Ta sẽ chứng minh đẳng thức sau bằng phương pháp quy nạp}
$$
A^n = \begin{pmatrix}
\cos n\theta & -\sin n\theta \\
\sin n\theta & \cos n\theta
\end{pmatrix}
$$
\text{với $n \in \mathbb{N}, n \ge 2$.}

\text{Trước hết, ta có}
$$
A^2 = A \cdot A = \begin{pmatrix}
\cos^2\theta - \sin^2\theta & -\cos\theta\sin\theta - \sin\theta\cos\theta \\
\sin\theta\cos\theta + \cos\theta\sin\theta & -\sin\theta\cos\theta + \cos^2\theta
\end{pmatrix}
= \begin{pmatrix}
\cos 2\theta & -\sin 2\theta \\
\sin 2\theta & \cos 2\theta
\end{pmatrix}
$$
\text{(Bước cơ sở đã được chứng minh cho $n=2$.)}

\text{Giả sử ta có đẳng thức đúng cho $n=k$, tức là}
$$
A^k = \begin{pmatrix}
\cos k\theta & -\sin k\theta \\
\sin k\theta & \cos k\theta
\end{pmatrix}
$$
\text{Ta sẽ chứng minh đẳng thức đúng cho $n=k+1$}
$$
A^{k+1} = \begin{pmatrix}
\cos (k+1)\theta & -\sin (k+1)\theta \\
\sin (k+1)\theta & \cos (k+1)\theta
\end{pmatrix}
$$

\text{Thật vậy, áp dụng giả thiết quy nạp và thực hiện phép nhân ma trận ta có}
$$
A^{k+1} = A^k \cdot A = \begin{pmatrix}
\cos k\theta & -\sin k\theta \\
\sin k\theta & \cos k\theta
\end{pmatrix}
\begin{pmatrix}
\cos\theta & -\sin\theta \\
\sin\theta & \cos\theta
\end{pmatrix}
$$
$$
= \begin{pmatrix}
\cos k\theta\cos\theta - \sin k\theta\sin\theta & -\cos k\theta\sin\theta - \sin k\theta\cos\theta \\
\sin k\theta\cos\theta + \cos k\theta\sin\theta & -\sin k\theta\sin\theta + \cos k\theta\cos\theta
\end{pmatrix}
$$
\text{Áp dụng công thức cộng lượng giác $\cos(A+B) = \cos A\cos B - \sin A\sin B$ và $\sin(A+B) = \sin A\cos B + \cos A\sin B$:}
$$
= \begin{pmatrix}
\cos (k\theta + \theta) & -\sin (k\theta + \theta) \\
\sin (k\theta + \theta) & \cos (k\theta + \theta)
\end{pmatrix}
= \begin{pmatrix}
\cos (k+1)\theta & -\sin (k+1)\theta \\
\sin (k+1)\theta & \cos (k+1)\theta
\end{pmatrix}
$$
\text{Vậy}
$$
A^n = \begin{pmatrix}
\cos n\theta & -\sin n\theta \\
\sin n\theta & \cos n\theta
\end{pmatrix}
$$

\end{document}
