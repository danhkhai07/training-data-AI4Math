\documentclass{article}
\usepackage[utf8]{inputenc}
\usepackage{amsmath, amssymb}

\begin{document}

\subsection*{Chứng minh rằng nếu $A$ là ma trận vuông thỏa mãn $A^n + A^{n-1} + \dots + A + I = O$ thì $A$ khả nghịch và $A^{-1} = A^n$.}

\section*{Giải.}

\text{Từ giả thiết ta có}
$$
A^n + A^{n-1} + \dots + A + I = O.
$$
\text{Ta nhân ma trận $A$ vào vế trái của biểu thức trên, ta được}
$$
A(A^n + A^{n-1} + \dots + A + I) = A \cdot O = O.
$$
\text{Khai triển vế trái:}
$$
A^{n+1} + A^n + \dots + A^2 + A = O.
$$
\text{Từ giả thiết $A^n + A^{n-1} + \dots + A + I = O$, ta có $A^n + A^{n-1} + \dots + A = -I$. }

\text{Thay $-I$ vào biểu thức $A^{n+1} + (A^n + \dots + A) = O$, ta có:}
$$
A^{n+1} + (-I) = O \quad \text{hay} \quad A^{n+1} = I.
$$
\text{Ta viết lại $A^{n+1} = I$ dưới dạng tích của $A$ và $A^n$:}
$$
A \cdot A^n = I.
$$
\text{Tương tự, ta có $A^n \cdot A = A^{n+1} = I$. }
\text{Theo định nghĩa ma trận nghịch đảo, $A$ khả nghịch và ma trận nghịch đảo của nó là $A^{-1} = A^n$.}

\end{document}
