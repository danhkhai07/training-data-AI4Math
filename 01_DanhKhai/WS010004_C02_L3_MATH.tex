\documentclass{article}
\usepackage[utf8]{inputenc}
\usepackage{amsmath, amssymb}
\usepackage{xfrac} 

\begin{document}

\subsection*{Bài 1.1.5. Cho ma trận $A = \begin{pmatrix} 1 & 1 & 0 \\ 0 & 1 & 1 \\ 0 & 0 & 1 \end{pmatrix}$ và ma trận $B = \begin{pmatrix} 0 & 1 & 0 \\ 0 & 0 & 1 \\ 0 & 0 & 0 \end{pmatrix}$.}
\begin{enumerate}
    \item[a)] Tính các ma trận $B^2, B^3$.
    \item[b)] Sử dụng các kết quả của câu a) hãy tính $A^n$, $n \in \mathbb{N}, n \ge 3$.
\end{enumerate}

\section*{Giải.}

\subsection*{a) Tính $B^2, B^3$}
\text{Ta có}
$$
B^2 = B \cdot B = \begin{pmatrix} 0 & 1 & 0 \\ 0 & 0 & 1 \\ 0 & 0 & 0 \end{pmatrix} \begin{pmatrix} 0 & 1 & 0 \\ 0 & 0 & 1 \\ 0 & 0 & 0 \end{pmatrix} = \begin{pmatrix} 0 & 0 & 1 \\ 0 & 0 & 0 \\ 0 & 0 & 0 \end{pmatrix}.
$$
$$
B^3 = B^2 \cdot B = \begin{pmatrix} 0 & 0 & 1 \\ 0 & 0 & 0 \\ 0 & 0 & 0 \end{pmatrix} \begin{pmatrix} 0 & 1 & 0 \\ 0 & 0 & 1 \\ 0 & 0 & 0 \end{pmatrix} = \begin{pmatrix} 0 & 0 & 0 \\ 0 & 0 & 0 \\ 0 & 0 & 0 \end{pmatrix} = O.
$$
\text{Suy ra $B^k = O, \forall k \ge 3$.}

\subsection*{b) Tính $A^n$}
\text{Ta có thể viết $A$ dưới dạng tổng của ma trận đơn vị $I$ và ma trận $B$:}
$$
A = \begin{pmatrix} 1 & 1 & 0 \\ 0 & 1 & 1 \\ 0 & 0 & 1 \end{pmatrix} = \begin{pmatrix} 1 & 0 & 0 \\ 0 & 1 & 0 \\ 0 & 0 & 1 \end{pmatrix} + \begin{pmatrix} 0 & 1 & 0 \\ 0 & 0 & 1 \\ 0 & 0 & 0 \end{pmatrix} = I + B.
$$
\text{Do $I \cdot B = B \cdot I = B$ (tức là $I$ và $B$ giao hoán) và $B^k = O, \forall k \ge 3$, ta áp dụng công thức nhị thức Newton cho ma trận:}
$$
A^n = (I + B)^n = \sum_{k=0}^{n} C_n^k I^{n-k} B^k = \sum_{k=0}^{n} C_n^k B^k.
$$
\text{Vì $B^k = O$ với mọi $k \ge 3$, nên tổng chỉ còn lại các số hạng với $k=0, 1, 2$:}
$$
A^n = C_n^0 I + C_n^1 B + C_n^2 B^2.
$$
\text{Ta có:}
\begin{itemize}
    \item $C_n^0 = 1$
    \item $C_n^1 = n$
    \item $C_n^2 = \dfrac{n(n-1)}{2}$
\end{itemize}
$$
A^n = I + nB + \dfrac{n(n-1)}{2} B^2
$$
$$
= \begin{pmatrix} 1 & 0 & 0 \\ 0 & 1 & 0 \\ 0 & 0 & 1 \end{pmatrix} + n \begin{pmatrix} 0 & 1 & 0 \\ 0 & 0 & 1 \\ 0 & 0 & 0 \end{pmatrix} + \dfrac{n(n-1)}{2} \begin{pmatrix} 0 & 0 & 1 \\ 0 & 0 & 0 \\ 0 & 0 & 0 \end{pmatrix}
$$
$$
= \begin{pmatrix}
1 & n & \dfrac{n(n-1)}{2} \\
0 & 1 & n \\
0 & 0 & 1
\end{pmatrix}.
$$

\end{document}
