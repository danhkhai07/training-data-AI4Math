\documentclass{article}
\usepackage[utf8]{inputenc}
\usepackage{amsmath, amssymb}
\usepackage{cases}

\begin{document}

\subsection*{Tìm chiều của không gian nghiệm của hệ phương trình}
$$
\begin{cases}
2x + 3t = 0 \\
-2y + z = 0
\end{cases}
$$

\section*{Giải.}

\text{Gọi } $A$ \text{ là không gian nghiệm của hệ phương trình thuần nhất đã cho.}
\text{Với mỗi } $\mathbf{u} = (x, y, z, t) \in A \text{ bất kỳ, khi đó}$
$$
\mathbf{u} = (-3C_2, C_1, 2C_1, 2C_2)
$$
$$
\mathbf{u} = C_1(0, 1, 2, 0) + C_2(-3, 0, 0, 2).
$$
\text{Đặt } $\mathbf{u}_1 = (0, 1, 2, 0), \mathbf{u}_2 = (-3, 0, 0, 2)$, \text{ ta suy ra } $\mathbf{u} = C_1\mathbf{u}_1 + C_2\mathbf{u}_2$. \text{Do đó } $\{\mathbf{u}_1, \mathbf{u}_2\}$ \text{ là một hệ sinh của } $A$.

\text{Mặt khác, } $\mathbf{u}_1, \mathbf{u}_2$ \text{ là hai véctơ không cùng phương nên chúng tạo thành hệ véctơ độc lập tuyến tính. Vậy } $\{\mathbf{u}_1, \mathbf{u}_2\}$ \text{ là một cơ sở của không gian } $A$ \text{ và } $\dim A = 2$.

\end{document}
