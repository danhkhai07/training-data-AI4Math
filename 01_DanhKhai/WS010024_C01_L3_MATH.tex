\documentclass{article}
\usepackage[utf8]{inputenc}
\usepackage{amsmath, amssymb}

\begin{document}

\subsection*{Tìm hạng của ma trận hệ số và hạng của ma trận hệ số mở rộng rồi tìm nghiệm tổng quát của hệ phương trình}
$$
\begin{cases}
2x_1 - x_2 + 3x_3 - 7x_4 = 5 \\
6x_1 - 3x_2 + x_3 - 4x_4 = 7 \\
4x_1 - 2x_2 + 14x_3 - 31x_4 = 18
\end{cases}
$$

\section*{Giải.}

\text{Sử dụng phương pháp Gauss, biến đổi ma trận hệ số mở rộng:}
$$
\bar{A} = \begin{pmatrix}
2 & -1 & 3 & -7 & | & 5 \\
6 & -3 & 1 & -4 & | & 7 \\
4 & -2 & 14 & -31 & | & 18
\end{pmatrix}
$$
\text{Thực hiện các phép biến đổi $\text{H}_2 \leftarrow \text{H}_2 - 3\text{H}_1$, $\text{H}_3 \leftarrow \text{H}_3 - 2\text{H}_1$:}
$$
\xrightarrow[\text{H}_3 \leftarrow \text{H}_3 - 2\text{H}_1]{\text{H}_2 \leftarrow \text{H}_2 - 3\text{H}_1}
\begin{pmatrix}
2 & -1 & 3 & -7 & | & 5 \\
0 & 0 & -8 & 17 & | & -8 \\
0 & 0 & 8 & -17 & | & 8
\end{pmatrix}
$$
\text{Thực hiện $\text{H}_3 \leftarrow \text{H}_3 + \text{H}_2$:}
$$
\xrightarrow{\text{H}_3 \leftarrow \text{H}_3 + \text{H}_2}
\begin{pmatrix}
2 & -1 & 3 & -7 & | & 5 \\
0 & 0 & -8 & 17 & | & -8 \\
0 & 0 & 0 & 0 & | & 0
\end{pmatrix}
$$
\text{Ma trận đã đạt dạng bậc thang. }
\text{Hạng của ma trận hệ số là $r(A) = 2$.}
\text{Hạng của ma trận hệ số mở rộng là $r(\bar{A}) = 2$.}
\text{Vì $r(A) = r(\bar{A}) = 2 < n = 4$ (số ẩn), hệ có vô số nghiệm phụ thuộc $4 - 2 = 2$ tham số.}

\text{Hệ phương trình tương đương với hệ sau:}
$$
\begin{cases}
2x_1 - x_2 + 3x_3 - 7x_4 = 5 \\
-8x_3 + 17x_4 = -8
\end{cases}
$$
\text{Chọn $x_2 = C_1$ và $x_4 = C_2$ là các tham số tùy ý ($C_1, C_2 \in \mathbb{R}$).}

\text{Từ phương trình thứ hai: }
$$
-8x_3 = 17x_4 - 8 = 17C_2 - 8
$$
$$
x_3 = \frac{8 - 17C_2}{8} = 1 - \frac{17}{8}C_2.
$$
\text{Thay $x_2, x_3, x_4$ vào phương trình thứ nhất:}
$$
2x_1 = 5 + x_2 - 3x_3 + 7x_4
$$
$$
2x_1 = 5 + C_1 - 3 \left(1 - \frac{17}{8}C_2\right) + 7C_2
$$
$$
2x_1 = 5 + C_1 - 3 + \frac{51}{8}C_2 + 7C_2
$$
$$
2x_1 = 2 + C_1 + \left(\frac{51}{8} + \frac{56}{8}\right)C_2 = 2 + C_1 + \frac{107}{8}C_2
$$
$$
x_1 = 1 + \frac{1}{2}C_1 + \frac{107}{16}C_2.
$$
\text{Vậy nghiệm tổng quát của hệ phương trình là:}
$$
\begin{cases}
x_1 = 1 + \frac{1}{2}C_1 + \frac{107}{16}C_2 \\
x_2 = C_1 \\
x_3 = 1 - \frac{17}{8}C_2 \\
x_4 = C_2
\end{cases}
$$
\text{với $C_1, C_2 \in \mathbb{R}$.}

\end{document}
