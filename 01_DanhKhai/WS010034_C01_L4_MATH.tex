\documentclass{article}
\usepackage[utf8]{inputenc}
\usepackage{amsmath, amssymb}

\begin{document}

\subsection*{Biện luận số nghiệm của hệ phương trình thuần nhất sau theo tham số $m$}
$$
\begin{cases}
(2m+1)x + y + z = 0 \\
x + (2m+1)y + z = 0 \\
x + y + (2m+1)z = 0
\end{cases}
$$

\section*{Giải.}

\text{Ta xét định thức của ma trận hệ số $A$:}
$$
\det A = \begin{vmatrix} 2m+1 & 1 & 1 \\ 1 & 2m+1 & 1 \\ 1 & 1 & 2m+1 \end{vmatrix}
$$
\text{Cộng cột 2 và cột 3 vào cột 1 ($\text{C}_1 \leftarrow \text{C}_1 + \text{C}_2 + \text{C}_3$):}
$$
\det A = \begin{vmatrix} 2m+3 & 1 & 1 \\ 2m+3 & 2m+1 & 1 \\ 2m+3 & 1 & 2m+1 \end{vmatrix}
$$
\text{Đưa nhân tử chung $(2m+3)$ ra ngoài:}
$$
\det A = (2m+3) \begin{vmatrix} 1 & 1 & 1 \\ 1 & 2m+1 & 1 \\ 1 & 1 & 2m+1 \end{vmatrix}
$$
\text{Thực hiện $\text{H}_2 \leftarrow \text{H}_2 - \text{H}_1$ và $\text{H}_3 \leftarrow \text{H}_3 - \text{H}_1$:}
$$
\det A = (2m+3) \begin{vmatrix} 1 & 1 & 1 \\ 0 & 2m & 0 \\ 0 & 0 & 2m \end{vmatrix}
$$
\text{Đây là định thức của ma trận tam giác, bằng tích các phần tử trên đường chéo chính:}
$$
\det A = (2m+3) \cdot 1 \cdot (2m) \cdot (2m) = 4m^2(2m+3).
$$

\subsection*{Biện luận}

\text{Vì đây là hệ phương trình thuần nhất, hệ luôn có nghiệm (nghiệm tầm thường $x=y=z=0$). Ta chỉ cần biện luận về nghiệm duy nhất hay vô số nghiệm.}

\subsubsection*{Trường hợp 1: $\det A \ne 0$}
\text{Hệ có nghiệm duy nhất (nghiệm tầm thường) khi $\det A \ne 0$, tức là:}
$$
4m^2(2m+3) \ne 0 \Leftrightarrow m \ne 0 \quad \text{và} \quad m \ne -\frac{3}{2}.
$$
\text{Khi đó, hệ có nghiệm duy nhất $x=y=z=0$.}

\subsubsection*{Trường hợp 2: $\det A = 0$}
\text{Hệ có vô số nghiệm khi $\det A = 0$, tức là $m = 0$ hoặc $m = -\frac{3}{2}$.}

\text{a) Khi $m = 0$:}
\text{Ma trận hệ số là $A = \begin{pmatrix} 1 & 1 & 1 \\ 1 & 1 & 1 \\ 1 & 1 & 1 \end{pmatrix}$. Hạng $r(A) = 1$.}
\text{Hệ tương đương $x + y + z = 0$. Hệ có vô số nghiệm phụ thuộc $3 - 1 = 2$ tham số.}
\text{Nghiệm tổng quát: $x = -C_1 - C_2, y = C_1, z = C_2$, với $C_1, C_2 \in \mathbb{R}$.}

\text{b) Khi $m = -\frac{3}{2}$:}
\text{Ma trận hệ số là $A = \begin{pmatrix} -2 & 1 & 1 \\ 1 & -2 & 1 \\ 1 & 1 & -2 \end{pmatrix}$.}
\text{Áp dụng phép biến đổi $\text{H}_1 \leftarrow \text{H}_1 + \text{H}_2 + \text{H}_3$: $\begin{pmatrix} 0 & 0 & 0 \\ 1 & -2 & 1 \\ 1 & 1 & -2 \end{pmatrix}$. Hạng $r(A) = 2$.}
\text{Hệ có vô số nghiệm phụ thuộc $3 - 2 = 1$ tham số.}
\text{Hệ tương đương $\begin{cases} x - 2y + z = 0 \\ 3y - 3z = 0 \end{cases} \Leftrightarrow \begin{cases} x = z \\ y = z \end{cases}$.}
\text{Nghiệm tổng quát: $x = C, y = C, z = C$, với $C \in \mathbb{R}$.}

\subsection*{Kết luận}
\begin{itemize}
    \item \text{Nếu $m \ne 0$ và $m \ne -\frac{3}{2}$: Hệ có nghiệm duy nhất $x=y=z=0$.}
    \item \text{Nếu $m = 0$: Hệ có vô số nghiệm phụ thuộc 2 tham số.}
    \item \text{Nếu $m = -\frac{3}{2}$: Hệ có vô số nghiệm phụ thuộc 1 tham số.}
\end{itemize}

\end{document}
