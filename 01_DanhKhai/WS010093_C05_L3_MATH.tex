\documentclass{article}
\usepackage[utf8]{inputenc}
\usepackage{amsmath, amssymb}
\usepackage{cases}

\begin{document}

\subsection*{Bài 3.1.20. Cho phép biến đổi tuyến tính $f: \mathbb{R}^2 \to \mathbb{R}^2, (x, y) \mapsto (4x + 2y, x + 5y)$. Tìm giá trị riêng và véctơ riêng của $f$.}

\section*{Giải.}

\text{Gọi } $E = \{\mathbf{e}_1 = (1, 0), \mathbf{e}_2 = (0, 1)\}$ \text{ là cơ sở chính tắc của } $\mathbb{R}^2$. \text{Ta có ma trận của } $f$ \text{ trong cơ sở chính tắc } $E$ \text{ là}
$$
A = \begin{pmatrix}
4 & 2 \\
1 & 5
\end{pmatrix}.
$$
\text{Đa thức đặc trưng của } $f$ \text{ là}
$$
P(\lambda) = \det(A - \lambda I) = \begin{vmatrix}
4 - \lambda & 2 \\
1 & 5 - \lambda
\end{vmatrix} = (4 - \lambda)(5 - \lambda) - 2 = \lambda^2 - 9\lambda + 18.
$$
\text{Giải phương trình đặc trưng } $P(\lambda) = 0$ \text{ ta thu được hai giá trị riêng } $\lambda_1 = 3, \lambda_2 = 6$.

\text{1. Với giá trị riêng } $\lambda_1 = 3$:
\text{Gọi } $(x, y)$ \text{ là tọa độ của véctơ riêng } $\mathbf{v}_1$ \text{ ứng với giá trị riêng } $\lambda_1 = 3$, \text{ thì } $(x, y)$ \text{ là nghiệm không tầm thường của hệ phương trình}
$$
(A - 3I) \begin{pmatrix} x \\ y \end{pmatrix} = \begin{pmatrix} 0 \\ 0 \end{pmatrix} \Leftrightarrow \begin{pmatrix}
4 - 3 & 2 \\
1 & 5 - 3
\end{pmatrix} \begin{pmatrix} x \\ y \end{pmatrix} = \begin{pmatrix} 0 \\ 0 \end{pmatrix}
$$
$$
\Leftrightarrow \begin{pmatrix}
1 & 2 \\
1 & 2
\end{pmatrix} \begin{pmatrix} x \\ y \end{pmatrix} = \begin{pmatrix} 0 \\ 0 \end{pmatrix} \Leftrightarrow x + 2y = 0 \Leftrightarrow x = -2y.
$$
\text{Chọn } $y = C \in \mathbb{R}$. \text{Vậy } $\mathbf{v}_1 = C(-2, 1), C \ne 0, C \in \mathbb{R}$ \text{ là véctơ riêng của } $f$ \text{ ứng với giá trị riêng } $\lambda_1 = 3$.

\text{2. Với giá trị riêng } $\lambda_2 = 6$:
\text{Gọi } $(x, y)$ \text{ là tọa độ của véctơ riêng } $\mathbf{v}_2$ \text{ ứng với giá trị riêng } $\lambda_2 = 6$, \text{ thì } $(x, y)$ \text{ là nghiệm không tầm thường của hệ phương trình}
$$
(A - 6I) \begin{pmatrix} x \\ y \end{pmatrix} = \begin{pmatrix} 0 \\ 0 \end{pmatrix} \Leftrightarrow \begin{pmatrix}
4 - 6 & 2 \\
1 & 5 - 6
\end{pmatrix} \begin{pmatrix} x \\ y \end{pmatrix} = \begin{pmatrix} 0 \\ 0 \end{pmatrix}
$$
$$
\Leftrightarrow \begin{pmatrix}
-2 & 2 \\
1 & -1
\end{pmatrix} \begin{pmatrix} x \\ y \end{pmatrix} = \begin{pmatrix} 0 \\ 0 \end{pmatrix} \Leftrightarrow x - y = 0 \Leftrightarrow x = y.
$$
\text{Chọn } $y = C \in \mathbb{R}$. \text{Vậy } $\mathbf{v}_2 = C(1, 1), C \ne 0, C \in \mathbb{R}$ \text{ là véctơ riêng của } $f$ \text{ ứng với giá trị riêng } $\lambda_2 = 6$.

\end{document}
