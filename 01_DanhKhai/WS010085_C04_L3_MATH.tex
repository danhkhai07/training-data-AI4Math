\documentclass{article}
\usepackage[utf8]{inputenc}
\usepackage{amsmath, amssymb}
\usepackage{cases}

\begin{document}

\subsection*{Bài 3.1.11. Cho $\varphi$ là một phép biến đổi tuyến tính trên $\mathbb{R}^3, (x, y, z) \mapsto (x - y + z, -y + z, x + 2y + 2z)$.}
\text{a) Tìm ma trận của $\varphi$ trong cơ sở chính tắc $E = \{\mathbf{e}_1, \mathbf{e}_2, \mathbf{e}_3\}$ của $\mathbb{R}^3$.} \\
\text{b) Chứng minh rằng hệ các véctơ $F = \{\mathbf{f}_1 = \mathbf{e}_1+\mathbf{e}_2, \mathbf{f}_2 = \mathbf{e}_2+\mathbf{e}_3, \mathbf{f}_3 = \mathbf{e}_3+\mathbf{e}_1\}$ là một cơ sở của $\mathbb{R}^3$. Tìm ma trận của $\varphi$ trong cơ sở $F$.} \\
\text{c) Tìm $\text{Im}\varphi$.}

\section*{Giải.}

\text{a) Ta có } $E = \{(1, 0, 0), (0, 1, 0), (0, 0, 1)\}$.
\text{Ta tìm ảnh của các véctơ cơ sở } $E$:
$$
\varphi(\mathbf{e}_1) = \varphi(1, 0, 0) = (1, 0, 1) = 1\mathbf{e}_1 + 0\mathbf{e}_2 + 1\mathbf{e}_3
$$
$$
\varphi(\mathbf{e}_2) = \varphi(0, 1, 0) = (-1, -1, 2) = -1\mathbf{e}_1 + (-1)\mathbf{e}_2 + 2\mathbf{e}_3
$$
$$
\varphi(\mathbf{e}_3) = \varphi(0, 0, 1) = (1, 1, 2) = 1\mathbf{e}_1 + 1\mathbf{e}_2 + 2\mathbf{e}_3
$$
\text{Nên ma trận của $\varphi$ trong cơ sở chính tắc $E$ là }
$$
A = [\varphi]_E = \begin{pmatrix}
1 & -1 & 1 \\
0 & -1 & 1 \\
1 & 2 & 2
\end{pmatrix}.
$$

\text{b) Ta có } $F = \{\mathbf{f}_1 = (1, 1, 0), \mathbf{f}_2 = (0, 1, 1), \mathbf{f}_3 = (1, 0, 1)\}$.
\text{Để chứng minh } $F$ \text{ là một cơ sở của không gian } $\mathbb{R}^3$, \text{ ta chỉ cần chứng minh } $\{\mathbf{f}_1, \mathbf{f}_2, \mathbf{f}_3\}$ \text{ là ba véctơ độc lập tuyến tính. Thật vậy, ta giải phương trình}
$$
\alpha\mathbf{f}_1 + \beta\mathbf{f}_2 + \gamma\mathbf{f}_3 = \mathbf{0} \Leftrightarrow \alpha(1, 1, 0) + \beta(0, 1, 1) + \gamma(1, 0, 1) = (0, 0, 0)
$$
$$
\Leftrightarrow
\begin{cases}
\alpha + \gamma = 0 \\
\alpha + \beta = 0 \\
\beta + \gamma = 0
\end{cases}
\Leftrightarrow \alpha = \beta = \gamma = 0.
$$
\text{Vậy } $\{\mathbf{f}_1, \mathbf{f}_2, \mathbf{f}_3\}$ \text{ độc lập tuyến tính hay } $F$ \text{ là một cơ sở của không gian } $\mathbb{R}^3$.

\text{Tìm ma trận của } $\varphi$ \text{ trong cơ sở } $F$. \text{Ta có } $[\varphi]_F = T_{F \leftarrow E} [\varphi]_E T_{E \leftarrow F}$.
\text{Ma trận chuyển cơ sở từ } $F$ \text{ sang } $E$ \text{ là } $T_{E \leftarrow F} = \begin{pmatrix} 1 & 0 & 1 \\ 1 & 1 & 0 \\ 0 & 1 & 1 \end{pmatrix}$.
\text{Ta tìm ảnh của các véctơ cơ sở } $F$:
$$
\varphi(\mathbf{f}_1) = \varphi(1, 1, 0) = (0, -1, 3)
$$
$$
\varphi(\mathbf{f}_2) = \varphi(0, 1, 1) = (0, 0, 4)
$$
$$
\varphi(\mathbf{f}_3) = \varphi(1, 0, 1) = (2, 1, 3)
$$
\text{Ta biểu diễn các ảnh này qua cơ sở } $F$:
\begin{align*}
\varphi(\mathbf{f}_1) &= a\mathbf{f}_1 + b\mathbf{f}_2 + c\mathbf{f}_3 \Leftrightarrow
\begin{cases}
a + c = 0 \\
a + b = -1 \\
b + c = 3
\end{cases}
\Leftrightarrow a=-2, b=1, c=2 \\
\varphi(\mathbf{f}_2) &= a\mathbf{f}_1 + b\mathbf{f}_2 + c\mathbf{f}_3 \Leftrightarrow
\begin{cases}
a + c = 0 \\
a + b = 0 \\
b + c = 4
\end{cases}
\Leftrightarrow a=-2, b=2, c=2 \\
\varphi(\mathbf{f}_3) &= a\mathbf{f}_1 + b\mathbf{f}_2 + c\mathbf{f}_3 \Leftrightarrow
\begin{cases}
a + c = 2 \\
a + b = 1 \\
b + c = 3
\end{cases}
\Leftrightarrow a=0, b=1, c=2
\end{align*}
\text{Nên ma trận của $\varphi$ trong cơ sở $F$ là }
$$
[\varphi]_F = \begin{pmatrix}
-2 & -2 & 0 \\
1 & 2 & 1 \\
2 & 2 & 2
\end{pmatrix}.
$$

\text{c) Ta tìm } $\text{Im}\varphi$. \text{Không gian ảnh } $\text{Im}\varphi$ \text{ sinh bởi ảnh của các véctơ cơ sở } $E$: $\{\varphi(\mathbf{e}_1), \varphi(\mathbf{e}_2), \varphi(\mathbf{e}_3)\}$.
\text{Xét ma trận tọa độ của các véctơ ảnh}
$$
A_{\text{Im}\varphi} = \begin{pmatrix}
1 & -1 & 1 \\
0 & -1 & 1 \\
1 & 2 & 2
\end{pmatrix}
\xrightarrow{H_3 \to H_3-H_1}
\begin{pmatrix}
1 & -1 & 1 \\
0 & -1 & 1 \\
0 & 3 & 1
\end{pmatrix}
\xrightarrow{H_3 \to H_3+3H_2}
\begin{pmatrix}
1 & -1 & 1 \\
0 & -1 & 1 \\
0 & 0 & 4
\end{pmatrix}
$$
\text{Ta thấy } $r(A_{\text{Im}\varphi}) = 3$. \text{Suy ra } $\dim \text{Im}\varphi = 3$. \text{Vì } $\text{Im}\varphi \triangleleft \mathbb{R}^3 \text{ và } \dim \text{Im}\varphi = 3$, \text{ nên } $\text{Im}\varphi = \mathbb{R}^3$.

\end{document}
