\documentclass{article}
\usepackage[utf8]{inputenc}
\usepackage{amsmath, amssymb}
\usepackage{cases}

\begin{document}

\subsection*{Bài 2.1.28. Hãy tìm chiều và chỉ ra một cơ sở của không gian véctơ sinh bởi các véctơ $\mathbf{a}_1 = (1, 0, 0, -1), \mathbf{a}_2 = (2, 1, 1, 0), \mathbf{a}_3 = (1, 1, 1, 1), \mathbf{a}_4 = (1, 2, 3, 4)$ và $\mathbf{a}_5 = (0, 1, 2, 3)$ trong $\mathbb{R}^4$.}

\section*{Giải.}

\text{Xét ma trận tọa độ của 5 véctơ trong cơ sở chính tắc của } $\mathbb{R}^4$.
$$
A = \begin{pmatrix}
1 & 2 & 1 & 1 & 0 \\
0 & 1 & 1 & 2 & 1 \\
0 & 1 & 1 & 3 & 2 \\
-1 & 0 & 1 & 4 & 3
\end{pmatrix}
$$
\text{Ta biến đổi sơ cấp về hàng của ma trận } $A$:
$$
A \xrightarrow{H_4 \to H_1+H_4}
\begin{pmatrix}
1 & 2 & 1 & 1 & 0 \\
0 & 1 & 1 & 2 & 1 \\
0 & 1 & 1 & 3 & 2 \\
0 & 2 & 2 & 5 & 3
\end{pmatrix}
$$
$$
\xrightarrow{H_3 \to -H_2+H_3}
\xrightarrow{H_4 \to -2H_2+H_4}
\begin{pmatrix}
1 & 2 & 1 & 1 & 0 \\
0 & 1 & 1 & 2 & 1 \\
0 & 0 & 0 & 1 & 1 \\
0 & 0 & 0 & 1 & 1
\end{pmatrix}
$$
$$
\xrightarrow{H_4 \to -H_3+H_4}
\begin{pmatrix}
1 & 2 & 1 & 1 & 0 \\
0 & 1 & 1 & 2 & 1 \\
0 & 0 & 0 & 1 & 1 \\
0 & 0 & 0 & 0 & 0
\end{pmatrix}
$$
\text{Ta suy ra } $r(A) = 3$. \text{Do đó } $\dim \mathcal{L}(\mathbf{a}_1, \mathbf{a}_2, \mathbf{a}_3, \mathbf{a}_4, \mathbf{a}_5) = 3$.

\text{Vì các cột thứ nhất, thứ hai và thứ tư chứa các véctơ cơ sở nên} $\{\mathbf{a}_1, \mathbf{a}_2, \mathbf{a}_4\}$ \text{ là một cơ sở của không gian } $\mathcal{L}(\mathbf{a}_1, \mathbf{a}_2, \mathbf{a}_3, \mathbf{a}_4, \mathbf{a}_5)$.

\end{document}
