\documentclass{article}
\usepackage[utf8]{inputenc}
\usepackage{amsmath, amssymb}
\usepackage{cases}

\begin{document}

\subsection*{Bài 2.1.30. Hãy tìm chiều và chỉ ra một cơ sở của không gian nghiệm của hệ phương trình}
$$
\begin{cases}
x_1 + x_2 + x_3 + x_4 + x_5 = 0 \\
x_1 - 2x_2 + 3x_3 - x_4 + 2x_5 = 0 \\
2x_1 - x_2 + 4x_3 + 3x_5 = 0
\end{cases}
$$

\section*{Giải.}

\text{Gọi } $B$ \text{ là không gian nghiệm của hệ phương trình thuần nhất đã cho. Để tìm một cơ sở của } $B$, \text{ ta cần tìm nghiệm tổng quát của hệ. Ta biến đổi ma trận hệ số}
$$
\begin{pmatrix}
1 & 1 & 1 & 1 & 1 \\
1 & -2 & 3 & -1 & 2 \\
2 & -1 & 4 & 0 & 3
\end{pmatrix}
\xrightarrow{H_2 \to -H_1+H_2}
\xrightarrow{H_3 \to -2H_1+H_3}
\begin{pmatrix}
1 & 1 & 1 & 1 & 1 \\
0 & -3 & 2 & -2 & 1 \\
0 & -3 & 2 & -2 & 1
\end{pmatrix}
$$
$$
\begin{pmatrix}
1 & 1 & 1 & 1 & 1 \\
0 & -3 & 2 & -2 & 1 \\
0 & -3 & 2 & -2 & 1
\end{pmatrix}
\xrightarrow{H_3 \to -H_2+H_3}
\begin{pmatrix}
1 & 1 & 1 & 1 & 1 \\
0 & -3 & 2 & -2 & 1 \\
0 & 0 & 0 & 0 & 0
\end{pmatrix}
$$
\text{Do đó hệ đã cho tương đương với}
$$
\begin{cases}
x_1 + x_2 + x_3 + x_4 + x_5 = 0 \\
-3x_2 + 2x_3 - 2x_4 + x_5 = 0
\end{cases}
$$
\text{Với mỗi } $\mathbf{u} = (x_1, x_2, x_3, x_4, x_5) \in B$ \text{ bất kỳ, ta có}
$$
\mathbf{u} = (-4x_2 + x_3 - 3x_4, x_2, x_3, x_4, 3x_2 - 2x_3 + 2x_4)
$$
$$
\mathbf{u} = x_2(-4, 1, 0, 0, 3) + x_3(1, 0, 1, 0, -2) + x_4(-3, 0, 0, 1, 2).
$$
\text{Đặt } $\mathbf{u}_1 = (-4, 1, 0, 0, 3), \mathbf{u}_2 = (1, 0, 1, 0, -2), \mathbf{u}_3 = (-3, 0, 0, 1, 2)$, \text{ ta suy ra } $\mathbf{u} = x_2\mathbf{u}_1 + x_3\mathbf{u}_2 + x_4\mathbf{u}_3$. \text{Do đó } $\{\mathbf{u}_1, \mathbf{u}_2, \mathbf{u}_3\}$ \text{ là hệ sinh của } $B$.

\text{Mặt khác, xét } $\alpha\mathbf{u}_1 + \beta\mathbf{u}_2 + \gamma\mathbf{u}_3 = \mathbf{0}$, \text{ hay } $\alpha(-4, 1, 0, 0, 3) + \beta(1, 0, 1, 0, -2) + \gamma(-3, 0, 0, 1, 2) = (0, 0, 0, 0, 0)$. \text{Ta thu được } $\alpha = \beta = \gamma = 0$ \text{ là nghiệm duy nhất của hệ. Do đó } $\{\mathbf{u}_1, \mathbf{u}_2, \mathbf{u}_3\}$ \text{ là một hệ véctơ độc lập tuyến tính.}

\text{Vậy } $\{\mathbf{u}_1, \mathbf{u}_2, \mathbf{u}_3\}$ \text{ là một cơ sở của không gian } $B$ \text{ và } $\dim B = 3$.

\end{document}
