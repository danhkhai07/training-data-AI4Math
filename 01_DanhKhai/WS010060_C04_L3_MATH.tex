\documentclass{article}
\usepackage[utf8]{inputenc}
\usepackage{amsmath, amssymb}
\usepackage{cases}

\begin{document}

\subsection*{Bài 2.1.27. Trong $\mathbb{R}^3$ cho hệ các véctơ}
$$
\mathbf{a}_1 = (-1, 3, 5), \mathbf{a}_2 = (2, 1, -3), \mathbf{a}_3 = (m, 2, -2), \mathbf{a}_4 = (1, m, 1).
$$
\text{a) Với } $m=2$, \text{ tìm chiều và một cơ sở của không gian con sinh bởi } $\{\mathbf{a}_1, \mathbf{a}_2, \mathbf{a}_3\}$. \\
\text{b) Tìm hạng của hệ véctơ } $\{\mathbf{a}_1, \mathbf{a}_2, \mathbf{a}_3, \mathbf{a}_4\}$ \text{ (theo } $m$).

\section*{Giải.}

\text{a) Với } $m=2$, \text{ ta có ma trận gồm tọa độ của các véctơ } $\{\mathbf{a}_1, \mathbf{a}_2, \mathbf{a}_3\}$ \text{ trong cơ sở chính tắc của } $\mathbb{R}^3$ \text{ là}
$$
A = \begin{pmatrix}
-1 & 2 & 2 \\
3 & 1 & 2 \\
5 & -3 & -2
\end{pmatrix}
$$
\text{Ta biến đổi sơ cấp về hàng của ma trận } $A$:
$$
A \xrightarrow{H_2 \to 3H_1+H_2}
\xrightarrow{H_3 \to 5H_1+H_3}
\begin{pmatrix}
-1 & 2 & 2 \\
0 & 7 & 8 \\
0 & 7 & 8
\end{pmatrix}
\xrightarrow{H_3 \to -H_2+H_3}
\begin{pmatrix}
-1 & 2 & 2 \\
0 & 7 & 8 \\
0 & 0 & 0
\end{pmatrix}
$$
\text{Ta thu được } $r(A) = 2$. \text{Do đó } $\dim \mathcal{L}(\mathbf{a}_1, \mathbf{a}_2, \mathbf{a}_3) = r(\mathbf{a}_1, \mathbf{a}_2, \mathbf{a}_3) = r(A) = 2$.
\text{Vì } $r(\mathbf{a}_1, \mathbf{a}_2) = 2$ \text{ nên } $\{\mathbf{a}_1, \mathbf{a}_2\}$ \text{ là một cơ sở của không gian } $\mathcal{L}(\mathbf{a}_1, \mathbf{a}_2, \mathbf{a}_3)$.

\text{b) Xét ma trận tọa độ của các véctơ } $\{\mathbf{a}_1, \mathbf{a}_2, \mathbf{a}_3, \mathbf{a}_4\}$ \text{ trong cơ sở chính tắc của } $\mathbb{R}^3$:
$$
B = \begin{pmatrix}
-1 & 2 & m & 1 \\
3 & 1 & 2 & m \\
5 & -3 & -2 & 1
\end{pmatrix}
$$
\text{Ta biến đổi sơ cấp về hàng của ma trận } $B$:
$$
B \xrightarrow{H_2 \to 3H_1+H_2}
\xrightarrow{H_3 \to 5H_1+H_3}
\begin{pmatrix}
-1 & 2 & m & 1 \\
0 & 7 & 3m+2 & m+3 \\
0 & 7 & 5m-2 & 6
\end{pmatrix}
$$
$$
\xrightarrow{H_3 \to -H_2+H_3}
\begin{pmatrix}
-1 & 2 & m & 1 \\
0 & 7 & 3m+2 & m+3 \\
0 & 0 & 2m-4 & -m+3
\end{pmatrix}
$$
\text{Ta thấy hai số } $2m-4$ \text{ và } $-m+3$ \text{ không đồng thời bằng } $0$ \text{ với mọi } $m \in \mathbb{R}$.
\begin{itemize}
    \item \text{Nếu } $2m-4 = 0 \Leftrightarrow m=2$, \text{ thì } $-m+3 = -2+3 = 1 \ne 0$.
    \item \text{Nếu } $-m+3 = 0 \Leftrightarrow m=3$, \text{ thì } $2m-4 = 2(3)-4 = 2 \ne 0$.
    \item \text{Nếu } $m \ne 2$ \text{ và } $m \ne 3$, \text{ thì cả hai đều khác } $0$.
\end{itemize}
\text{Do đó, luôn có một hàng cơ sở chứa phần tử khác } $0$ \text{ ở hàng thứ ba.}

\text{Vậy hạng của hệ véctơ } $\{\mathbf{a}_1, \mathbf{a}_2, \mathbf{a}_3, \mathbf{a}_4\}$ \text{ bằng } $3$ \text{ với mọi } $m \in \mathbb{R}$.

\end{document}
