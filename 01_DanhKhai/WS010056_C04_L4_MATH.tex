\documentclass{article}
\usepackage[utf8]{inputenc}
\usepackage{amsmath, amssymb}

\begin{document}

\subsection*{Bài 2.1.23. Chứng minh rằng hệ đa thức $F = \{1, x-\alpha, (x-\alpha)^2, \dots, (x-\alpha)^n\}$, với $\alpha \in \mathbb{R}$ cố định, là một cơ sở của $P_n[x]$. Tìm tọa độ của $p(x) = x^2+2x+3$ trong cơ sở $F$.}

\section*{Giải.}

\text{Vì } $\dim P_n[x] = n+1$ \text{ nên để chứng minh hệ vectơ } $F$ \text{ là một cơ sở của không gian } $P_n[x]$ \text{ ta chỉ cần chứng minh } $F$ \text{ là một hệ sinh của không gian } $P_n[x]$.

\text{Thật vậy, với mỗi } $q(x) \in P_n[x]$ \text{ bất kỳ, ta có } $q(x)$ \text{ là đa thức bậc nhỏ hơn hoặc bằng } $n$ \text{ nên khả vi mọi cấp trên } $\mathbb{R}$. \text{Theo công thức khai triển Taylor của } $q(x)$ \text{ tại } $x = \alpha$, \text{ ta được}
$$
q(x) = q(\alpha) + \frac{q'(\alpha)}{1!}(x-\alpha) + \frac{q''(\alpha)}{2!}(x-\alpha)^2 + \dots + \frac{q^{(n)}(\alpha)}{n!}(x-\alpha)^n \quad (*)
$$
\text{Hay } $F$ \text{ là một hệ sinh của không gian } $P_n[x]$. \text{ Vậy } $F$ \text{ là một cơ sở của không gian } $P_n[x]$.

\begin{itemize}
    \item \text{Tìm tọa độ của } $p(x) = x^2+2x+3$ \text{ trong cơ sở } $F$.
\end{itemize}

\text{Ta có } $p(x) = x^2+2x+3$. \text{ Ta tính các đạo hàm tại } $x=\alpha$:
$$
p'(x) = 2x + 2, \quad p''(\alpha) = 2, \quad p^{(k)}(x) = 0, \forall k > 2.
$$
\text{Áp dụng công thức } $(*)$, \text{ ta được}
\begin{align*}
p(\alpha) &= \alpha^2 + 2\alpha + 3 \\
p'(\alpha) &= 2\alpha + 2 \\
p''(\alpha) &= 2 \\
p^{(k)}(\alpha) &= 0, \forall k > 2.
\end{align*}
\text{Do đó, khai triển Taylor của } $p(x)$ \text{ tại } $\alpha$ \text{ là}
$$
p(x) = (\alpha^2+2\alpha+3) \cdot 1 + \frac{(2\alpha+2)}{1!}(x-\alpha) + \frac{2}{2!}(x-\alpha)^2 + 0 \cdot (x-\alpha)^3 + \dots
$$
$$
p(x) = (\alpha^2+2\alpha+3) \cdot 1 + (2\alpha+2)(x-\alpha) + 1 \cdot (x-\alpha)^2.
$$
\text{Do đó tọa độ của } $p(x)$ \text{ trong cơ sở } $F$ \text{ là } $\left( \alpha^2+2\alpha+3, 2\alpha+2, 1, 0, \dots, 0 \right)$.

\end{document}
