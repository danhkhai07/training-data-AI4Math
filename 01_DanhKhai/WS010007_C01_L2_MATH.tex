\documentclass{article}
\usepackage[utf8]{inputenc}
\usepackage{amsmath, amssymb}

\begin{document}

\subsection*{Giải hệ phương trình tuyến tính sau}
$$
\begin{cases}
2x_1 + x_2 + 3x_3 + 2x_4 = 5 \\
4x_1 + 3x_2 + 5x_3 + x_4 = 7 \\
6x_1 + 5x_2 + 7x_3 + 4x_4 = 1
\end{cases}
$$

\section*{Giải.}

\text{Ma trận mở rộng của hệ là:}
$$
\bar{A} = \begin{pmatrix}
2 & 1 & 3 & 2 & | & 5 \\
4 & 3 & 5 & 1 & | & 7 \\
6 & 5 & 7 & 4 & | & 1
\end{pmatrix}
$$

\text{Ta thực hiện các phép biến đổi sơ cấp trên hàng để đưa ma trận về dạng bậc thang:}
$$
\bar{A} \xrightarrow[\text{H}_3 \leftarrow \text{H}_3 - 3\text{H}_1]{\text{H}_2 \leftarrow \text{H}_2 - 2\text{H}_1}
\begin{pmatrix}
2 & 1 & 3 & 2 & | & 5 \\
0 & 1 & -1 & -3 & | & -3 \\
0 & 2 & -2 & -2 & | & -14
\end{pmatrix}
$$
$$
\xrightarrow{\text{H}_3 \leftarrow \text{H}_3 - 2\text{H}_2}
\begin{pmatrix}
2 & 1 & 3 & 2 & | & 5 \\
0 & 1 & -1 & -3 & | & -3 \\
0 & 0 & 0 & 4 & | & -8
\end{pmatrix}
$$

\text{Từ ma trận bậc thang này, ta xác định hạng của ma trận hệ số ($r(A)$) và hạng của ma trận mở rộng ($r(\bar{A})$):}
$$
\text{Hạng } r(A) = 3 \quad \text{và} \quad \text{Hạng } r(\bar{A}) = 3.
$$
\text{Số ẩn là } $n = 4$.

\text{Vì $r(A) = r(\bar{A}) < n$ nên hệ phương trình có vô số nghiệm phụ thuộc vào $n - r(A) = 4 - 3 = 1$ tham số.}

\text{Hệ phương trình tương đương với:}
$$
\begin{cases}
2x_1 + x_2 + 3x_3 + 2x_4 = 5 \\
x_2 - x_3 - 3x_4 = -3 \\
4x_4 = -8
\end{cases}
$$

\text{Giải ngược từ dưới lên:}
\begin{itemize}
    \item $4x_4 = -8 \Rightarrow x_4 = -2$.
    \item Chọn $x_3 = \alpha$ là tham số tùy ý ($\alpha \in \mathbb{R}$).
    \item Thay $x_3 = \alpha$ và $x_4 = -2$ vào phương trình thứ hai:
    $$
    x_2 - \alpha - 3(-2) = -3 \Rightarrow x_2 - \alpha + 6 = -3 \Rightarrow x_2 = \alpha - 9.
    $$
    \item Thay $x_2 = \alpha - 9$, $x_3 = \alpha$, $x_4 = -2$ vào phương trình thứ nhất:
    $$
    2x_1 + (\alpha - 9) + 3\alpha + 2(-2) = 5
    $$
    $$
    2x_1 + 4\alpha - 13 = 5 \Rightarrow 2x_1 = 18 - 4\alpha \Rightarrow x_1 = 9 - 2\alpha.
    $$
\end{itemize}

\text{Vậy nghiệm của hệ phương trình là:}
$$
\begin{pmatrix} x_1 \\ x_2 \\ x_3 \\ x_4 \end{pmatrix}
= \begin{pmatrix} 9 - 2\alpha \\ \alpha - 9 \\ \alpha \\ -2 \end{pmatrix}
= \begin{pmatrix} 9 \\ -9 \\ 0 \\ -2 \end{pmatrix} + \alpha \begin{pmatrix} -2 \\ 1 \\ 1 \\ 0 \end{pmatrix}, \quad \text{với } \alpha \in \mathbb{R}.
$$

\end{document}
