\documentclass{article}
\usepackage[utf8]{inputenc}
\usepackage{amsmath, amssymb}
\usepackage{cases}

\begin{document}

\subsection*{Bài 3.1.5. Cho hai phép biến đổi tuyến tính $u, v: \mathbb{R}^2 \to \mathbb{R}^2$, biết ma trận của $u$ trong cơ sở chính tắc của $\mathbb{R}^2$ là $A = \begin{pmatrix} 2 & -2 \\ 1 & 3 \end{pmatrix}$ và ma trận của $v$ trong cơ sở $B = \{(2, 1), (1, 1)\}$ là $B' = \begin{pmatrix} 1 & 2 \\ 1 & -1 \end{pmatrix}$. Tìm ma trận của $h = u \circ v$ trong cơ sở chính tắc của $\mathbb{R}^2$.}

\section*{Giải.}

\text{Ma trận của ánh xạ hợp $h = u \circ v$ trong cơ sở chính tắc $E$ của $\mathbb{R}^2$ là } $[h]_E = [u]_E [v]_E$. \\
\text{Ta có } $[u]_E = \begin{pmatrix} 2 & -2 \\ 1 & 3 \end{pmatrix}$. \text{Ta cần tìm } $[v]_E$.

\text{Ma trận chuyển cơ sở từ cơ sở chính tắc $E = \{(1, 0), (0, 1)\}$ sang cơ sở $B = \{(2, 1), (1, 1)\}$ là }
$$
T_{B \leftarrow E} = (T_{E \leftarrow B})^{-1}.
$$
\text{Ma trận chuyển cơ sở từ $B$ sang $E$ là } $T_{E \leftarrow B} = \begin{pmatrix} 2 & 1 \\ 1 & 1 \end{pmatrix}$.
\text{Ta tính ma trận nghịch đảo } $T_{B \leftarrow E}$:
$$
T_{B \leftarrow E} = (T_{E \leftarrow B})^{-1} = \frac{1}{\det(T_{E \leftarrow B})} \begin{pmatrix} 1 & -1 \\ -1 & 2 \end{pmatrix} = \frac{1}{2\cdot 1 - 1\cdot 1} \begin{pmatrix} 1 & -1 \\ -1 & 2 \end{pmatrix} = \begin{pmatrix} 1 & -1 \\ -1 & 2 \end{pmatrix}.
$$
\text{Ma trận của phép biến đổi tuyến tính $v$ trong cơ sở chính tắc là}
$$
[v]_E = T_{E \leftarrow B} [v]_B T_{B \leftarrow E} = \begin{pmatrix} 2 & 1 \\ 1 & 1 \end{pmatrix} \begin{pmatrix} 1 & 2 \\ 1 & -1 \end{pmatrix} \begin{pmatrix} 1 & -1 \\ -1 & 2 \end{pmatrix}
$$
$$
= \begin{pmatrix} 3 & 3 \\ 2 & 1 \end{pmatrix} \begin{pmatrix} 1 & -1 \\ -1 & 2 \end{pmatrix} = \begin{pmatrix} 3(1) + 3(-1) & 3(-1) + 3(2) \\ 2(1) + 1(-1) & 2(-1) + 1(2) \end{pmatrix} = \begin{pmatrix} 0 & 3 \\ 1 & 0 \end{pmatrix}.
$$
\text{Vậy ma trận của $h = u \circ v$ trong cơ sở chính tắc của $\mathbb{R}^2$ là}
$$
[h]_E = [u]_E [v]_E = \begin{pmatrix} 2 & -2 \\ 1 & 3 \end{pmatrix} \begin{pmatrix} 0 & 3 \\ 1 & 0 \end{pmatrix} = \begin{pmatrix} 2(0) + (-2)(1) & 2(3) + (-2)(0) \\ 1(0) + 3(1) & 1(3) + 3(0) \end{pmatrix} = \begin{pmatrix} -2 & 6 \\ 3 & 3 \end{pmatrix}.
$$

\end{document}
