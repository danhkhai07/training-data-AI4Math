\documentclass{article}
\usepackage[utf8]{inputenc}
\usepackage{amsmath, amssymb}
\usepackage{cases}

\begin{document}

\subsection*{Bài 3.1.15. Cho phép biến đổi tuyến tính $f: \mathbb{R}^3 \to \mathbb{R}^3$. Biết $f(1, 1, 0) = (1, 1, 0), f(1, 0, 1) = (1, 0, 1), f(0, 1, 1) = (2, 2, 2)$.}
\text{a) Chứng minh $F = \{\mathbf{f}_1 = (1, 1, 0), \mathbf{f}_2 = (1, 0, 1), \mathbf{f}_3 = (0, 1, 1)\}$ là một cơ sở của $\mathbb{R}^3$.} \\
\text{b) Tìm ma trận của $f$ trong cơ sở chính tắc $E$ của $\mathbb{R}^3$.} \\
\text{c) Tìm ma trận của $f$ trong cơ sở $F$.}

\section*{Giải.}

\text{a) Xét ma trận chuyển cơ sở từ $F$ sang $E$: }
$$
T_{E \leftarrow F} = \begin{pmatrix}
1 & 1 & 0 \\
1 & 0 & 1 \\
0 & 1 & 1
\end{pmatrix}.
$$
\text{Ta tính định thức:}
$$
\det(T_{E \leftarrow F}) = 1(0\cdot 1 - 1\cdot 1) - 1(1\cdot 1 - 1\cdot 0) + 0(1\cdot 1 - 0\cdot 0) = -1 - 1 = -2.
$$
\text{Vì } $\det(T_{E \leftarrow F}) = -2 \ne 0$, \text{ nên } $F$ \text{ là hệ véctơ độc lập tuyến tính. Do } $\dim \mathbb{R}^3 = 3$ \text{ nên } $F$ \text{ là một cơ sở của } $\mathbb{R}^3$.

\text{b) Ta tìm ma trận của $f$ trong cơ sở chính tắc $E$ là } $A = [f]_E = \begin{pmatrix} [f(\mathbf{e}_1)]_E & [f(\mathbf{e}_2)]_E & [f(\mathbf{e}_3)]_E \end{pmatrix}$.
\text{Ta biểu diễn các véctơ cơ sở chính tắc } $\mathbf{e}_i$ \text{ qua cơ sở } $F$. \text{Từ hệ}
$$
\mathbf{x} = x_1\mathbf{f}_1 + x_2\mathbf{f}_2 + x_3\mathbf{f}_3 \Leftrightarrow
\begin{cases}
x_1 + x_2 = x \\
x_1 + x_3 = y \\
x_2 + x_3 = z
\end{cases}
\Leftrightarrow
\begin{cases}
x_1 = \frac{x+y-z}{2} \\
x_2 = \frac{x-y+z}{2} \\
x_3 = \frac{-x+y+z}{2}
\end{cases}
$$
\text{Áp dụng công thức cho } $\mathbf{e}_1 = (1, 0, 0), \mathbf{e}_2 = (0, 1, 0), \mathbf{e}_3 = (0, 0, 1)$:
$$
\mathbf{e}_1 = \frac{1}{2}\mathbf{f}_1 + \frac{1}{2}\mathbf{f}_2 - \frac{1}{2}\mathbf{f}_3
$$
$$
\mathbf{e}_2 = \frac{1}{2}\mathbf{f}_1 - \frac{1}{2}\mathbf{f}_2 + \frac{1}{2}\mathbf{f}_3
$$
$$
\mathbf{e}_3 = -\frac{1}{2}\mathbf{f}_1 + \frac{1}{2}\mathbf{f}_2 + \frac{1}{2}\mathbf{f}_3
$$
\text{Ta có } $f(\mathbf{f}_1) = \mathbf{f}_1, f(\mathbf{f}_2) = \mathbf{f}_2, f(\mathbf{f}_3) = 2\mathbf{f}_3$. \text{Do đó}
\begin{align*}
f(\mathbf{e}_1) &= \frac{1}{2}f(\mathbf{f}_1) + \frac{1}{2}f(\mathbf{f}_2) - \frac{1}{2}f(\mathbf{f}_3) = \frac{1}{2}\mathbf{f}_1 + \frac{1}{2}\mathbf{f}_2 - \frac{1}{2}(2\mathbf{f}_3) = \frac{1}{2}\mathbf{f}_1 + \frac{1}{2}\mathbf{f}_2 - \mathbf{f}_3 \\
&= \frac{1}{2}(1, 1, 0) + \frac{1}{2}(1, 0, 1) - (0, 1, 1) = \left( \frac{1}{2}+\frac{1}{2}, \frac{1}{2}+0-1, 0+\frac{1}{2}-1 \right) = (1, -\frac{1}{2}, -\frac{1}{2}).
\end{align*}
\begin{align*}
f(\mathbf{e}_2) &= \frac{1}{2}f(\mathbf{f}_1) - \frac{1}{2}f(\mathbf{f}_2) + \frac{1}{2}f(\mathbf{f}_3) = \frac{1}{2}\mathbf{f}_1 - \frac{1}{2}\mathbf{f}_2 + \frac{1}{2}(2\mathbf{f}_3) = \frac{1}{2}\mathbf{f}_1 - \frac{1}{2}\mathbf{f}_2 + \mathbf{f}_3 \\
&= \frac{1}{2}(1, 1, 0) - \frac{1}{2}(1, 0, 1) + (0, 1, 1) = \left( \frac{1}{2}-\frac{1}{2}, \frac{1}{2}-0+1, 0-\frac{1}{2}+1 \right) = (0, \frac{3}{2}, \frac{1}{2}).
\end{align*}
\begin{align*}
f(\mathbf{e}_3) &= -\frac{1}{2}f(\mathbf{f}_1) + \frac{1}{2}f(\mathbf{f}_2) + \frac{1}{2}f(\mathbf{f}_3) = -\frac{1}{2}\mathbf{f}_1 + \frac{1}{2}\mathbf{f}_2 + \frac{1}{2}(2\mathbf{f}_3) = -\frac{1}{2}\mathbf{f}_1 + \frac{1}{2}\mathbf{f}_2 + \mathbf{f}_3 \\
&= -\frac{1}{2}(1, 1, 0) + \frac{1}{2}(1, 0, 1) + (0, 1, 1) = \left( -\frac{1}{2}+\frac{1}{2}, -\frac{1}{2}+0+1, 0+\frac{1}{2}+1 \right) = (0, \frac{1}{2}, \frac{3}{2}).
\end{align*}
\text{Ma trận của $f$ trong cơ sở chính tắc $E$ là}
$$
A = [f]_E = \begin{pmatrix}
1 & 0 & 0 \\
-1/2 & 3/2 & 1/2 \\
-1/2 & 1/2 & 3/2
\end{pmatrix} = \frac{1}{2} \begin{pmatrix}
2 & 0 & 0 \\
-1 & 3 & 1 \\
-1 & 1 & 3
\end{pmatrix}.
$$

\text{c) Ma trận của $f$ trong cơ sở $F$ là } $[f]_F = \begin{pmatrix} [f(\mathbf{f}_1)]_F & [f(\mathbf{f}_2)]_F & [f(\mathbf{f}_3)]_F \end{pmatrix}$.
\text{Ta có}
$$
f(\mathbf{f}_1) = \mathbf{f}_1 = 1\mathbf{f}_1 + 0\mathbf{f}_2 + 0\mathbf{f}_3
$$
$$
f(\mathbf{f}_2) = \mathbf{f}_2 = 0\mathbf{f}_1 + 1\mathbf{f}_2 + 0\mathbf{f}_3
$$
$$
f(\mathbf{f}_3) = (2, 2, 2) = 2\mathbf{f}_3 = 0\mathbf{f}_1 + 0\mathbf{f}_2 + 2\mathbf{f}_3
$$
\text{Vậy ma trận của $f$ trong cơ sở $F$ là}
$$
[f]_F = \begin{pmatrix}
1 & 0 & 0 \\
0 & 1 & 0 \\
0 & 0 & 2
\end{pmatrix}.
$$

\end{document}
