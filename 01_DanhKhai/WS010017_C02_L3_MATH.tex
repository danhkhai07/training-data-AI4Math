\documentclass{article}
\usepackage[utf8]{inputenc}
\usepackage{amsmath, amssymb}

\begin{document}

\subsection*{Biện luận hạng của ma trận sau theo tham số $m$:}
$$
A = \begin{pmatrix}
1 & -1 & 2 & 1 \\
2 & -2 & m+5 & m^2+1 \\
1 & -1 & 2 & m-1
\end{pmatrix}
$$

\section*{Giải.}

\text{Dùng phép biến đổi sơ cấp về hàng đưa ma trận đã cho về dạng bậc thang:}
$$
\begin{pmatrix}
1 & -1 & 2 & 1 \\
2 & -2 & m+5 & m^2+1 \\
1 & -1 & 2 & m-1
\end{pmatrix}
\xrightarrow[\text{H}_3 \leftarrow \text{H}_3 - \text{H}_1]{\text{H}_2 \leftarrow \text{H}_2 - 2\text{H}_1}
\begin{pmatrix}
1 & -1 & 2 & 1 \\
0 & 0 & m+1 & m^2-1 \\
0 & 0 & 0 & m-2
\end{pmatrix}
$$

\text{Ma trận đã cho về dạng bậc thang. Hạng của ma trận $A$ bằng số hàng khác không trong ma trận này.}
\begin{itemize}
    \item \text{Nếu $m+1 = 0$ (tức $m=-1$) và $m-2 = 0$ (tức $m=2$):}
    \text{Xét trường hợp $m=-1$:}
    $$
    \begin{pmatrix}
    1 & -1 & 2 & 1 \\
    0 & 0 & 0 & 0 \\
    0 & 0 & 0 & -3
    \end{pmatrix} \quad \text{Hạng } r(A) = 2.
    $$
    \text{Xét trường hợp $m=2$:}
    $$
    \begin{pmatrix}
    1 & -1 & 2 & 1 \\
    0 & 0 & 3 & 3 \\
    0 & 0 & 0 & 0
    \end{pmatrix} \quad \text{Hạng } r(A) = 2.
    $$
    \text{Vậy, nếu $m = -1$ hoặc $m = 2$ thì $r(A) = 2$.}
    
    \item \text{Nếu $m \notin \{-1; 2\}$:}
    \text{Cả hai phần tử $m+1$ và $m-2$ đều khác 0. Ma trận bậc thang có 3 hàng khác không.}
    $$
    r(A) = 3.
    $$
\end{itemize}

\end{document}
