\documentclass{article}
\usepackage[utf8]{inputenc}
\usepackage{amsmath, amssymb}

\begin{document}

\subsection*{Bài 3.1.29. Cho phép biến đổi tuyến tính $T: P_n[x] \to P_n[x], p(x) \mapsto p'(x)$, ($p'(x)$ là hàm đạo hàm của $p(x)$). Tìm các giá trị riêng và các vectơ riêng tương ứng của $T$.}

\section*{Giải.}
\text{Từ Bài 3.1.12, ta có ma trận của $T$ trong cơ sở chính tắc là}
$$
A = \begin{pmatrix}
0 & 1 & 0 & \dots & 0 \\
0 & 0 & 2 & \dots & 0 \\
0 & 0 & 0 & \dots & 0 \\
\vdots & \vdots & \vdots & \ddots & \vdots \\
0 & 0 & 0 & \dots & n \\
0 & 0 & 0 & \dots & 0
\end{pmatrix}.
$$
\text{Từ đó suy ra đa thức đặc trưng của phép biến đổi tuyến tính $T$ là}
$$
\det(A - \lambda I) = (-\lambda)^n.
$$
\text{Vậy $\lambda = 0$ là giá trị riêng duy nhất của phép biến đổi tuyến tính $T$.}

\text{Vectơ riêng tương ứng của $T$ là nghiệm của phương trình $T(p(x)) = 0 \cdot p(x)$ tức là $T(p(x)) = 0$. Mặt khác, do $T(p(x)) = p'(x)$ nên $p'(x) = 0$, suy ra $p(x) = C$ (với $C$ là hằng số).}

\text{Vậy $p(x) = C, C \ne 0$ là vectơ riêng tương ứng với giá trị riêng duy nhất $\lambda = 0$ của phép biến đổi tuyến tính $T$.}

\end{document}
